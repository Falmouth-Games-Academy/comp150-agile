\documentclass{scrartcl}

\usepackage[hidelinks]{hyperref}
\usepackage[none]{hyphenat}

\title{Does the Improved Agile Team Cohesion from Pair Programming Affect Software Quality?}
\subtitle{COMP150 - Agile Essay Draft Assignment}

\author{Samantha Wills}

\begin{document}

\maketitle

\section*{Introduction}
In the games industry, the quality of the software can be effected and interpreted in different ways. For many, it is the functional cohesion of the software which constitutes as high quality while others include code comprehension and commenting. During the process of programming, there can be a variety of factors influencing the programmers ability to code and can often change the speed and quality. Therefore, in many agile practices, pair programming is used to reduce common mistakes and focus code direction. This essay aims to investigate the ways pair programming can be used and has proven to improve standards of work. It also aims to investigate the effect of team strength and relationship of the programming pair in their attempt to improve software quality.

\section*{Main Body}
Agile practises are often used in game development studio to allow for group discussion and improve retrospective evaluation of current projects to improve project quality. For many companies, the cost of quality assurance and testers can prove costly during and post production. If pair programming can be used to improve software quality then it could reduce the costs of fixing bugs and player approval.

The definition of software quality differs depending on the individual and their role within the game industry. A programmer would perceive code commenting, naming conventions and data structure. For this essay, the measurement of software quality is the number of errors found and the qualitative description of game software written. The comparison should be two groups working on the same task: pair programmers and individuals.



\bibliographystyle{ieeetran}
\bibliography{comp150_agile}

\end{document}
