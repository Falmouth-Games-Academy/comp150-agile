\documentclass{beamer}
\usetheme{Boadilla}

\begin{document}
	\small
	\begin{frame}
	
	\section{Agile software testing in a large-scale project}
	
	[1] Agile testing implemented on a large project designed for military use. Results showed the team cut by an order of magnitude the time required to fix defects, defect longevity, and defect-management overhead. Implementing the agile process required a new mind-set at personal and organizational levels, and it remains a challenge to expand its adoption to other projects in the organisation
	
	\section{Communication between Developers and Testers in Distributed Continuous Agile Testing}
	
	[2] Four main types of communication between tester and developer;
	handover through issue tracker system
	formal meetings
	written communication 
	coordination by mutual adjustment
	Early participation of the testers is very important to the success of the handover between testers and developers accomplished through participation in planning, standups and review meetings. Communication is not sufficiently effective through written communication and it must be augmented by informal communication.	
	\end{frame}
	\begin{frame}

	\section{Games User Research (GUR) for Indie Studios }
	
	[4] For playtesting if the participants selected do not effectively represent the developers target audience, then the findings cannot be confidently applied to the game. If teams could independently develop persona's for their games they will have an easier time finding representative participants for their user tests. The precision of recorded observational data allows developers to examine user behaviour as a complete sequence of moments in time. When developers are present during playtests and directly witness players' experiences, they become more motivated to fix issues immediately and note what players enjoyed to provide similar experiences in the future.
	\end{frame}
	\begin{frame}
	\section{Using prototypes in early pervasive game development}
	
	[3] We also discussed using “real” players and professional test players. Players who belong to the target group of the game usually provide more relevant data and are useful for understanding the players’ attitudes, opinions, and behavior. Using professional test players (e.g., colleagues) for testing enables faster iteration, and is also beneficial when new ideas are needed. If the prototype is very incomplete, it can be difficult for outsiders to understand, so it is useful to have both kinds of test players in the same project.
	
	\section{I have no words & I must design}
	
	[5] Game design is ultimately a process of iterative refinement, continuous adjustment during testing, until, budget and schedule and management willing, we have a polished product that does indeed work beautifully, wonderfully, superbly. But your changes of getting that beautiful, wonderful, superb game will be much higher if you begin with intentionality, begin by thinking about the experiences you want your players to have, understand what makes a game, and understand what pleasures people find in them.
	\end{frame}
	\begin{frame}
	\small	[1]D. Talby, A. Keren, O. Hazzan and Y. Dubinsky, "Agile software testing in a large-scale project", IEEE Software, vol. 23, no. 4, pp. 30-37, 2006.
		
		[2]D. Cruzes, N. Moe and T. Dyba, "Communication between Developers and Testers in Distributed Continuous Agile Testing", 2016 IEEE 11th International Conference on Global Software Engineering (ICGSE), 2016. 
		
		[3]E. Ollila, R. Suomela and J. Holopainen, "Using prototypes in early pervasive game development", Computers in Entertainment, vol. 6, no. 2, p. 1, 2008.
		
		[4]N. Moosajee and P. Mirza-Babaei, "Games User Research (GUR) for Indie Studios", Proceedings of the 2016 CHI Conference Extended Abstracts on Human Factors in Computing Systems - CHI EA '16, 2016.
		
		[5]G. Costikyan, "I have no words & I must design: Toward a critical vocabulary for games.", F. Mdiyrd (Ed.), Proceedings of the Computer Games and Digital Cultures Conference, 2002.
	\end{frame}
\end{document}