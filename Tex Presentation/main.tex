\documentclass{beamer}
\usetheme{Boadilla}

\title{Research Findings Presentation}

\begin{document}

\begin{frame}
\titlepage
\end{frame}

\begin{frame}
\frametitle{Outline}
\tableofcontents
\end{frame}

\section{Sam}
\subsection{My research question}
\begin{frame}
\frametitle{My research question}
My first research question: Why is agile so popular within the industry and is scrum the best methodology.
\\~\\
My final research question: How long should a sprint last when developing games?\\
\begin{itemize}
\item Comparing sprint lengths
\item Time management
\item Game development work flow
\item content quantity vs quality for different sprint lengths
\end{itemize}
\end{frame}

\subsection{References}
\begin{frame}
\frametitle{References}
\begin{itemize}
\item Towards Agent-based Agile Approach for Game
Development Methodology 
\item Impact and Challenges of Requirements Elicitation and Prioritization in Quality to Agile Process: Scrum as
a Case Scenario
\item Are the Old Days Gone? A Survey on Actual Software
Engineering Processes in Video Game Industry
\item Risk Management in Video Game Development Projects
\item What Concerns Game Developers?
A Study on Game Development Processes, Sustainability and Metrics
\item The Risks of Agile Software Development
\end{itemize}
\end{frame}

\subsection{My research findings}
\begin{frame}
Getting a good time frame for sprints is crucial for getting consistent work done when developing games as it sets up a work flow for developers to work with.\\
\begin{itemize}
\frametitle{My research findings}
\item Sprint lengths usually depend on the scale of the project
\item 1 to 2 week sprints are more common within the games industry
\item Teams doing longer sprints usually end up doing more crunch time
\end{itemize}
But I think 1 week sprints are the ideal time frame for agile based games development, as it makes sprints more focused on creating new content to be implemented within a game.
\end{frame}

\section{Adrian}
\subsection{My research question}
\begin{frame}
\frametitle{My research question}
Is early and frequent playtesting and prototyping as used in the agile philosophy beneficial to the development process or is it a time sink compared with late stage Quality Assurance testing as used in the Waterfall approach?
\end{frame}

\subsection{Research Findings}
\begin{frame}
\frametitle{My Research Findings}
Research Paper - Continuous and Collaborative Validation: A Field Study of Requirements Knowledge in Agile 
\\http://ieeexplore.ieee.org/document/5457339/\\~\\

''Iterations promote conversations among stakeholders and the development team…Test-driven development (TDD) promotes early communication among testers, programmer, and Product Owner.''
\\~\\

This paper promotes the idea that testing throughout the development process can benefit team communication.
\end{frame}

\begin{frame}
\frametitle{}
Research Paper - Costs of compliance: agile in an inelastic organization
\\http://ieeexplore.ieee.org/document/1609822/
\\~\\
''There was true collaboration, helped considerably by the test-driven approach which guaranteed that the test team received code of good quality. The test team could then focus on the operational and deployment issues associated with releasing a product, rather than catching the development team’s mistakes. In the end, together the team delivered a product that had two orders of magnitude fewer defects than other products in the division.''
\\~\\
Like the last paper, this also supports the idea that testing early and often influences the productivity of the development team as a whole, essentially streamlining development.
\end{frame}

\begin{frame}
\frametitle{}
Research Paper - Towards Agent-based Agile approach for Game Development Methodology
\\http://ieeexplore.ieee.org/stamp/stamp.jsp?arnumber=6916626\\~\\

''Game development is not a linear process. Iteration is the life of game development. Game developers use Waterfall methodology with enhancements, by adding iteration to the methodology.''
\\~\\
Evidence suggesting that even games developed with the waterfall approach should use the idea of iteration proposed in agile philosophy.
\end{frame}

\begin{frame}
\frametitle{}
Research Paper - Comparing extreme programming and Waterfall project results
\\http://ieeexplore.ieee.org/document/5876129/\\~\\

''Initially the faculty thought that Test-Driven Development would increase the amount of testing code, however, given a slow adoption rate of Test-Driven Development, programmers resorted to what was familiar and thus produced similar results.''
\\~\\
Interestingly the findings of this paper suggest very little difference in overall results between Waterfall and Agile. It does suggest however this may be due too poor implementation of the Test-Driven Method.
\end{frame}

\begin{frame}
\frametitle{}
Research Paper - Are the Old Days Gone? A Survey on Actual Software Engineering Processes in Video Game Industry
\\http://ieeexplore.ieee.org/document/7809512/\\~\\

''...we found that iterative processes are actually mainstream in video game industry and agile practices adoption is increasing in the last years''
''...iterative practices are increasing their adoption and are applied in at least 55 percent of projects. Moreover, agile practices are explicitly adopted in 45 percent of projects. Waterfall process is still applied in 30 percent of the projects.''
\\~\\
This paper talks about how Agile is being used more and more in the games industry but not exclusively, some game companies still prefer the more traditional and rigid Waterfall methodology. This paper was particularly useful in that it had references to other games that used either the iterative Agile approach or the more stoic Waterfall approach.
\end{frame}

\begin{frame}
\frametitle{}
Postmortem: Kingdoms of Amalur: Reckoning – Agile\\

This article supports the idea that regular consumer testing is beneficial when frequently implemented throughout the development process. There is also evidence that shows how frequent iteration and playtesting can help to mitigate the need for a large crunch at the end of development as would happen using Waterfall, bugs and design flaws can be spotted and fixed much sooner.
\\~\\
Postmortem: Frozenbyte's Trine – Waterfall\\

This article explains how testing of the game went wrong with reference to a lack of early playtesting and the author clearly supports the idea that the waterfall style can hinder development when testing is done later rather than sooner.
\end{frame}

\begin{frame}
\frametitle{}
Postmortem: Unknown Worlds Entertainment's Natural Selection 2 – Agile\\
This article advocates the use of early playtesting, done both by the team and consumers. With the team playing the game so often it makes it difficult to ignore gameplay problems. The author also comments on how important feedback from non-designers was through each iteration of the game.
\\~\\
Postmortem: Moon Studios' heartfelt Ori and the Blind Forest – Agile\\
"Creating an early prototype that was focused to show the entire team AND our publisher that we’ve got a really hot iron in the fire proved extremely useful. After we had something insanely fun to play, we had this intrinsic feeling that we could definitely pull it off and Microsoft was able to just sit down and play what we had created and understand that we weren’t just daydreaming."
\end{frame}

\subsection{Conclusion}
\begin{frame}
\frametitle{Conclusion}
Most of these Post-mortems are useful as they are evidential of games that have actually gone through the development process, however I am aware that it is largely anecdotal and the views of the developers may be biased or inaccurate.\\ Having said that, I think my research generally supports my own view that early and regular prototyping and playtesting is better that just post-production Quality Assurance.
\end{frame}

\section{Nathan}
\subsection{Research Question}
\begin{frame}
\frametitle{Research Question}
If teams used task boards does team work increase?
\end{frame}

\subsection{References}
\begin{frame}
\frametitle{Reference Papers}

\end{frame}

\subsection{Research Findings}
\begin{frame}
\frametitle{Research Findings}
a great way to manage workloads for teams\\
 must not be used by itself and must be paired with else in agile\\
 allow the project owner to see where everybody\\
 provides a real time view into what work is remaining, what is in progress and what have been completed\\
 some teams color code the post-its to distinguish which group need to do what for example blue might be for programmers while pink might be for artist\\
\end{frame}

\section{Darren}
\subsection{Research Question}
\begin{frame}
\frametitle{Research Question}
Is the use of the Agile philosophy as beneficial as its perceived to be?
\end{frame}
\subsection{My research findings}
\begin{frame}
\frametitle{Problems within Agile}
A few of the biggest problems within the games industry is time management meeting pre-agreed deadlines and budgeting. According to Standish Group's CHAOS report in 2015 19 percent of projects failed or challenged (52 percent) so the need of a productive and time efficient methodology is crucial. \\~\\

The Agile philosophy all be it flexible has got problems as it relies heavily on scrums, stand-ups, and the constant understanding of what to do now and then later. A heavy problem with this approach is how reliant the philosophy is on productive members of the team and the team members being able to communicate freely without any issue.\\~\\

Other issues that arise is the lack of motivation with doing sprints and stand-ups, this issue could be due to them being boring by nature and that by this leads to less work being done and less communication.
\end{frame}

\subsection{References}
\begin{frame}
\frametitle{Reference Papers}
\begin{itemize}
\item Gamifying software development scrum project IEEE
\\(http://ieeexplore.ieee.org.ezproxy.falmouth.ac.uk/document/8056584/)
\item Problems in the Adoption of Agile-Scrum Methodologies: A Systematic Literature Review\\ (http://ieeexplore.ieee.org.ezproxy.falmouth.ac.uk/document/7477924/)
\item A hybrid approach to solve the agile team allocation problem\\(http://ieeexplore.ieee.org.ezproxy.falmouth.ac.uk/document/6252999/)
\item Introducing agile methods: three years of experience\\(http://ieeexplore.ieee.org.ezproxy.falmouth.ac.uk/document/1333388/)
\item Applying Agile Methodologies in Industry Projects: Benefits and Challenges\\(http://ieeexplore.ieee.org.ezproxy.falmouth.ac.uk/document/7155964/)
\item Lehman's laws in agile and non-agile projects\\(http://ieeexplore.ieee.org.ezproxy.falmouth.ac.uk/document/6671304/)
\end{itemize}
\end{frame}

\end{document}