% Please do not change the document class
\documentclass{scrartcl}

% Please do not change these packages
\usepackage[hidelinks]{hyperref}
\usepackage[none]{hyphenat}
\usepackage{setspace}
\doublespace

% You may add additional packages here
\usepackage{amsmath}

% Please include a clear, concise, and descriptive title
\title{How Project Managers Can Adapt To Using Agile In Games Development}

% Please do not change the subtitle
\subtitle{COMP150 - Agile Essay}

% Please put your student number in the author field
\author{1506530}

\begin{document}

\maketitle

\abstract{In game development environments, leadership is required for the management of different activities in different situations e.g, in the case of leading partially distributed teams, in the case of coordinating large scale development, or in the case of collaborating with testing managers\cite{Leader}. In Agile, the leadership structure seems to support an equal status approach where everyone has a voice and a choice. Each team has a Scrum Master (SM), who has no real authority over the team, only the process that can be executed. It can be a difficult transition over to Agile, from the traditional Command and Control structure. The SM can change the process to resolve an issue i.e. if a sprint was not fully complete. The SM can resolve this by asking “What can be done to fix this?”, but can NOT force anyone to solve it. Traditionally a quick overview of what went wrong should be enough information for the SM to realise how a solution can be applied and a direct handing out of jobs to the right people would solve the issue. Much like a producer would in a games development studio. How would the industry go about implementing SM's instead of project managers in the games industry? That is what this paper intends on finding out.}

\section{What Is Agile?}
 The Agile Manifesto claims that the methodology is followed using the following: 
 \begin{center}
 	Individuals and interactions over processes and tools
 	\newline
 	Working software over comprehensive documentation
 	\newline
 	Customer collaboration over contract negotiation
 	\newline
 	Responding to change over following a plan
 	\newline
 	\newline
 	That is, while there is value in the items on
 	\newline
 	the right, we value the items on the left more\cite{manifesto}.
 	\newline
 \end{center}
 
It is argued that relying on the left more than the right will cause chaotic and undisciplined software\cite{ready,Chaos}. It is also said that Agile Software Development methods were originally considered only successful for small projects\cite{ready,extreme}. Though a projects output produces a higher rate of viable and working software when adopting the agile method, plus an improved responsiveness to the customers needs which result in enhanced software quality\cite{LeaderUnleashed,MasterActivities}. Evidence suggests that implementation into larger projects can be difficult\cite{MasterActivities}. This difficulty does not originate from the technological limitations but from the social and organisational limits\cite{SocialAgile}. The evidence provided states that this limit can be overcome by a competent SM with a deep understanding of Agile\cite{Together}. There are many procedures that can be exploited to achieve success. Although in this paper it will be concluded that agile can be integrated into the games industry, a discussion will be made on how this can be done through the use of Scrum Masters (SM's). Points will be made on how a Project Manager (PM) should change their leadership style to more suit Agile. 

\section{Management}


\subsection{Project Managers}
 
 Project Managers (PM's) are considered to be solely responsible for a projects success or failure\cite{Behavior}. To manage this responsibility it is widely accepted that the PM use a command and control style of management\cite{MasterActivities}. PM's are required for all projects across the world, they see to the administrative and organisational requirements of the project\cite{Oxymoron}. This can be difficult to bring together in an environment where the PM is to blame when things go wrong. Therefore to be a PM one need to have perspective, nerve, communication and creativity the usual requirements to being a manager\cite{Oxymoron}. When Agile is introduced to a project there can be great resistance from the PM's. As they are used to seeing themselves as the authority figure trying to keep all the power to themselves it is difficult for a PM to understand the Agile method\cite{Leader,ready}, even with strong evidence to support Agile and its benefits.
 
\subsection{Scrum Masters}

It is essential to have scrum masters for a project to be fully integrated into the agile philosophy\cite{Together}. The obvious first choice for the SM position would be the PM's, as they are already in a position of authority. As tradition dictates PM's tend to use a command and control method\cite{ManagerMaster}, it can be difficult for the PM's to give up this authority\cite{Together}. In agile a common myth is conceived that a self-organised team does not need direction, if this were the case there would be a significant sustainability issue\cite{Oxymoron,Together}. To overcome this potential issue a Scrum Master (SM) is used instead of a PM to guide the team\cite{ManagerMaster}.The majority of PM's will defend the theory that Agile creates more chaos and will tend to leave\cite{Together,ready}, on the other hand some will argue that agile will help productivity. These are the PM's that have the minds set to be a SM. Although to achieve this productivity and integrate Agile correctly the PM's will have to be adopt the correct leadership methods.

\subsection{Implementation}

Software development like in the games industry, need good leaders for the teams to work effectively, these leader will deal with the administrative workload while coaching their teams. Sometimes this workload will include some coordination of efforts, obtaining resources and aligning people to the standards that are required\cite{Oxymoron,Together,Behavior,ManagerMaster}. Evidence suggests that to change the mindset of a PM to become a SM, a few simple guidelines should be followed:
\begin{center}
	More people initiative and less top-down control.
		\newline
	More team players and less individual heroes.
		\newline
	More courage and less risk avoidance.
		\newline
	More conversations and less one-way communication.
		\newline
	More personal growth and less comfort zone\cite{AdoptAgile}.
		\newline
\end{center}

Though these may be controversial to the command and control structure that PM's are used too, these guidelines are a good start for new SM's to become effective leaders for their teams\cite{Leader,AdoptAgile}. As the SM's are in essence a leader for their team and not a manager, PM's with greater people skills would be more suited for the SM position\cite{Together}. There is a risk in having a PM become an SM: as the people around them may be affected in their self-managing decision making from the pressure of having their old PM still with them\cite{ManagerMaster}. Due to this risk an approach that has been widely taken on-board, the PM can not become a SM to the team directly under their line of authority\cite{ManagerMaster,Together}. This has had positive effects in the companies that have adopted Agile in this way, though the work may be more chaotic\cite{Chaos}, the results to support its success\cite{LeaderUnleashed}.

\section{Conclusion}

In conclusion to this paper it seems that though it may be difficult to implement Agile into the games industry, i.e. swapping and changing the PM's around so that the self-management is not impeded by their continued presence amongst their current line of management. It seems that PM's do bare value when becoming SM's as they already have the experience necessary to adequately lead their teams and guide them through each project. Their new self-organising nature will allow the SM to concentrate on customer relations and project yield.

\bibliographystyle{ieeetr}
\bibliography{COMP150_Agile_Essay}

\end{document}
