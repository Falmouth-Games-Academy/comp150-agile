% Please do not change the document class
\documentclass{scrartcl}

% Please do not change these packages
\usepackage[hidelinks]{hyperref}
\usepackage[none]{hyphenat}
\usepackage{setspace}
\doublespace

% You may add additional packages here
\usepackage{amsmath}

% Please include a clear, concise, and descriptive title
\title{How Project Managers Can Adapt To Using Agile In Games Development}

% Please do not change the subtitle
\subtitle{COMP150 - Agile Essay}

% Please put your student number in the author field
\author{1506530}

\begin{document}

\maketitle

\abstract {In game development, it can be difficult to implement a new style of management use couching and support with their teams. Each team must have a Scrum Master (SM), who has no real authority over the team, only the process that can be executed. It can be a difficult transition over to Agile and coaching, from the traditional Command and Control style. The SM can change the process to resolve an issue i.e. if a sprint was not fully complete. An SM however would be able to change the process of resolving an issue but can not manage the team. Traditionally a quick overview of an issue would be enough information for the PM to realise how a solution can be applied and a direct handing out of jobs to the right people would solve the issue. Much like a producer would in a games development studio.}

\section{What Is Agile?}
 
Looking at the Agile manifesto, it is argued that relying on the left more than the right will cause chaotic and undisciplined software\cite{ready,Chaos}. It is also said that Agile Software Development methods were originally considered only successful for small projects\cite{ready,manifesto,extreme}. The output of a project results in a higher rate of viable and working software when adopting the agile method, additionally an improved responsiveness to the customers needs which result in enhanced software quality\cite{LeaderUnleashed,MasterActivities}. Evidence suggests that implementation into larger projects can be difficult but possible\cite{MasterActivities}. This difficulty does not originate from the technological limitations but from the social and organisational limits\cite{SocialAgile}. These limits tend to stem from the PM's unwillingness to change from command and control to couching and support. The evidence provided suggests that this limit can be overcome by a change in management style and a temporary SM with a deep understanding of Agile\cite{Together}. Although this paper supports that using a PM as an SM is acceptable, but also the Agile team still needs managing through their self organising. A discussion will be made on how why this is. 

\section{Management}


\subsection{Project Managers}
 
 Project Managers (PM's) are considered to be solely responsible for a projects success or failure\cite{Behavior}. To manage this responsibility conventional wisdom recommends that the PM use a command and control style of management\cite{MasterActivities}. PM's are required for all projects across the world, they see to the administrative and organisational requirements of the project\cite{Oxymoron}. This can be difficult to bring together in an environment where the PM is to blame when things do not go to plan. Therefore to be a PM one need to have perspective, nerve, communication and creativity: the customary requirements to being a manager\cite{Oxymoron}. When Agile is introduced to a project there can be great resistance from some PM's. As they are used to seeing themselves as the sole authority figure trying to keep all the power to themselves. It is difficult for a PM to convert to the Agile methodology\cite{Leader,ready}, even with strong evidence to support Agile and its benefits.
 
\subsection{Scrum Masters}

It is essential to have scrum masters for a project to be fully integrated into the agile philosophy\cite{Together}. A potential choice for the SM position would be the PM's, as they are already in a position of authority and have experience in people management. As tradition supports PM's tend to use a command and control style of management\cite{ManagerMaster}, it can be difficult for the PM's to surrender style of management\cite{Together}. A common myth has been conceived that a self-organised team does not need direction, if this were the case there would be future sustainability issues\cite{Oxymoron,Together}. To overcome these potential issues, a Scrum Master (SM) is used instead of a PM to guide and coach the team\cite{ManagerMaster}.The majority of PM's will argue that Agile creates more chaos, and these PM's tend to leave the company\cite{Together,ready}, on the other side there is the argument that agile helps productivity. As the SM's are in essence a coach for their team and not a manager, PM's with greater people skills would be more suited for the SM position\cite{Together}. Although to achieve this productivity and integrate Agile correctly the PM's will have to be adopt the correct coaching methodologies.

\subsection{Implementation}

Due to the multidisciplinary nature of the games industry there is a need for effective SM's for teams to work efficiently, these leader will deal with the administrative workload while coaching their teams. Sometimes this workload will include some coordination of efforts, obtaining resources and aligning people to the standards that are required\cite{Oxymoron,Together,Behavior,ManagerMaster}. Evidence suggests that to change the mindset of a command and control PM to become a coaching and supportive SM, a few simple guidelines should be followed:
\begin{center}
	More people initiative and less top-down control.
		\newline
	More team players and less individual heroes.
		\newline
	More courage and less risk avoidance.
		\newline
	More conversations and less one-way communication.
		\newline
	More personal growth and less comfort zone\cite{AdoptAgile}.
		\newline
\end{center}

Though these may be controversial to the command and control structure that traditional PM's are used to: these guidelines are a good start for new SM's to become effective coaches and provide support for their teams\cite{Leader,AdoptAgile}. There is a risk in having a traditional command and control PM become an SM: as the people directly below their management line may be affected in their self-organising decision making. This originates from the pressure of having their old PM still with them\cite{ManagerMaster}. Due to this risk an approach that has been effectively implemented, would recommend: that the PM can not become a SM to the team directly under their line of authority\cite{ManagerMaster,Together}. This implementation has had positive effects on the companies that have adopted Agile in this way. Though the work may be more chaotic\cite{Chaos}, the results to support its success\cite{LeaderUnleashed}.

\section{Conclusion}

In conclusion to this paper it seems that in some cases it may be difficult to implement Agile into the games industry, i.e. swapping and changing the PM's around so that the self-management is not impeded by their continued presence amongst their current line of management. It seems that command and control PM's do poses value when becoming SM's as they already have adequate experience necessary to adapt to their teams needs and coach them through each project. Their new self-organising nature will also allow the SM to concentrate on customer relations and project yield.

\bibliographystyle{ieeetr}
\bibliography{COMP150_Agile_Essay}

\end{document}
