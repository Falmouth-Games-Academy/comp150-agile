% Please do not change the document class
\documentclass{scrartcl}

% Please do not change these packages
\usepackage[hidelinks]{hyperref}
\usepackage[none]{hyphenat}
\usepackage{setspace}
\doublespace

% You may add additional packages here
\usepackage{amsmath}

% Please include a clear, concise, and descriptive title
\title{How Project Managers Could Adapt To Using Agile For Games Production}

% Please do not change the subtitle
\subtitle{COMP150 - Agile Essay}

% Please put your student number in the author field
\author{1506530}

\begin{document}

\maketitle

\abstract {In game development, it can be difficult to implement a new style of management. In Agile each team must have a Scrum Master (SM), this SM has command and control authority over the team, only the process that something can be done. The SM can change this process to resolve an issue i.e. if a sprint was not fully complete, however the SM can not manage the team and command them through each task. Due to this lack of authority it can be a difficult for a Project Manager (PM) transition over to Agile and coaching, from the Traditional Command and Control (TCC) style. In this paper a recommendation is made on a successful method of transitioning over to Agile using a traditional command and control PM.}
\newpage

\section{What Is Agile?}

 The Agile Manifesto claims that the methodology is followed using the following: 
 \begin{center}
 	``Individuals and interactions over processes and tools
 	\newline
 	Working software over comprehensive documentation
 	\newline
 	Customer collaboration over contract negotiation
 	\newline
 	Responding to change over following a plan
 	\newline
 	That is, while there is value in the items on
 	\newline
 	the right, we value the items on the left more\cite{manifesto}.''
 	\newline
 \end{center}
 
An argument had arisen that relying on the ``left'' more than the ``right'' will cause chaotic and undisciplined software\cite{ready,Chaos}. Although the output of a project results in a higher rate of viable and working software when adopting the agile method, additionally an improved responsiveness to the customers needs which result in enhanced software quality\cite{LeaderUnleashed,MasterActivities}. It is also argued that Agile Software Development methods were considered only successful for small projects for instance an Indie Games Project\cite{ready,extreme}. Evidence suggests that implementation into larger projects can be difficult but possible\cite{MasterActivities}. This difficulty does not originate from the technological limitations but from the social and organisational limits\cite{SocialAgile}. These limits tend to stem from the PM's unwillingness to change from command and control to coaching and support. The evidence provided suggests that this limit can be overcome by replacing some PM's and hiring temporary SM's with a deep understanding of Agile\cite{Together}. This paper argues that keeping TCC PM's as SM's is recommended, due to their management experience. Transitioning over to Agile in larger projects should be a smoother transition when using a PM as an SM. A discussion will be made on why TCC PM's are valuable commodities in the games industry and a recommendation for a method of transition into Agile, for larger projects is made. 

\section{Management}


\subsection{Managers And Masters}
 
Project Managers (PM's) are considered to be solely responsible for a projects success or failure\cite{Behavior}. To manage this responsibility conventional wisdom dictates that the PM use a command and control style of management\cite{MasterActivities}. PM's are required for all projects across the games industry, they see to the administrative and organisational requirements of the project\cite{Oxymoron}. This can be difficult to bring together in an environment where the PM is to blame when things do not go to plan. Therefore, PM's need to have perspective, nerve, communication and creativity\cite{Oxymoron}. When Agile is introduced to a project there can be resistance from some PM's. This is due to seeing themselves as the sole authority figure over the project team. Even with suggested evidence stating that an SM ``protects'' the project team from the project owner by commanding the flow of information, which in essence is relatively the same as holding the responsibility for the whole project. It can still be difficult for PM's to embrace the Agile principle\cite{Leader,ready} as all the PM's see are the downsides to Agile\cite{ready}.
\newline
 
It is essential however, for a project to have scrum masters to fully integrated the Agile philosophy\cite{Together}. A potential choice for the SM position would be the PM's, as they have experience in people management. Although as tradition dictates, PM's should use the command and control style of management\cite{ManagerMaster}. Some PM's will argue that Agile creates more chaos, and these PM's tend to leave the company\cite{Together,ready}, on the other side there is the argument that agile helps productivity. As the SM's are also a coach for their team and not a manager, PM's with greater people management skills would be more suited for the SM position\cite{Together}. To achieve this productivity and integrate Agile correctly, the PM's will have to be adopt the correct coaching methodologies.

\subsection{Implementation}

 Evidence suggests that to change the mindset of a TCC PM to become a coaching and supportive SM, a few simple guidelines should be followed:
 \begin{center}
 	``More people initiative and less top-down control.
 	\newline
 	More team players and less individual heroes.
 	\newline
 	More courage and less risk avoidance.
 	\newline
 	More conversations and less one-way communication.
 	\newline
 	More personal growth and less comfort zone\cite{AdoptAgile}.''
 	\newline
 \end{center}

 Though these may be controversial to the command and control structure that traditional PM's are used to, these guidelines are a good start for new SM's to become effective coaches and provide support for their teams\cite{Leader,AdoptAgile}. There is a risk in having a TCC PM become an SM, as the multidisciplinary nature of the games industry requires effective SM's for teams to work efficiently. Evidence suggests that issues have arisen from the use of PM's as SM's, these issues are caused if managers retain their command and control over their line of management. Which in turn cause sustainability issues\cite{Together,Oxymoron}. as the people directly below their management line may be affected in their self-organising decision making. This stems from the pressure of having their old PM still within the project team\cite{ManagerMaster}. An approach that has been effectively implemented would recommend that to resolve this issue, the PM's can not become an SM to the team directly under their line of authority\cite{ManagerMaster,Together}. This implementation has had positive effects on the companies that have adopted Agile in this way. Though the work may be more chaotic\cite{Chaos}, the results do support its success\cite{LeaderUnleashed,Together}. 
 
\section{Conclusion}

Having a PM's that can successfully transfer to Agile will benefit the project teams, as they are already familiar with the administrative duties entailed within a project. A TCC PM as a SM can deal with the administrative workload while coaching and guiding their teams. Though in some cases it may be difficult to implement Agile into the games industry, i.e. swapping and changing the PM's around so that the self-management is not impeded by their continued presence amongst their current line of management, it is recommended by this paper that TCC PM's do posses value when becoming SM's. Their position and skills should be carefully contemplated before removing or adapting them, as their new self-organising nature will also allow the PM now SM, to concentrate on administration and customer relations. To which they are already familiar and have experience in.

\bibliographystyle{ieeetr}
\bibliography{COMP150_Agile_Essay}

\end{document}
