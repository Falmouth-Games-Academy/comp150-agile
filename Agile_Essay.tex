% Please do not change the document class
\documentclass{scrartcl}

% Please do not change these packages
\usepackage[hidelinks]{hyperref}
\usepackage[none]{hyphenat}
\usepackage{setspace}
\doublespace

% You may add additional packages here
\usepackage{amsmath}

\title{How can introverts best adapt to the agile philosophy?}

\subtitle{COMP150 - Agile Development Practice}

\author{1608351}

\begin{document}

\maketitle

\abstract{A variety of personalities and skill-sets in individuals within a team tends to enhance performance, therefore it is important not to disregard those who struggle with agile principles. Finding methods that give individuals the autonomy to align their working style with agile, may help ensure diversity within teams.  This paper explores how individuals, with personality types that have been found to have a negative relationship with agile preference, can best adapt to the agile philosophy.}

\section{Introduction}

Agile processes are widely used in the Game Development industry, they encourage a focus on self-organising teams along with regular reflection, and communication \cite{Agile}. Due to the scope involved in game development tasks, working within teams, rather than individually, is necessary. However, such focus on teamwork and communication can be problematic for introverts. A correlation has been found between an individual's personality type, based on the Five-Factor model (FFM), and their preference for agile processes \textit{`The data supports a positive relationship between extraversion and agile preference as well as openness and agile preference'} \cite{BishopDeokar}. FFM marks personality traits across a continuum of five distinct strands, namely: Extraversion, Openness, Agreeableness, Neuroticism, and Conscientiousness. Although individuals are rarely able to be placed in one category exclusively, there is usually a tendency towards certain factors in FFM, and therefore, this tool can still be useful in managing effects of personality on the workplace.

Subjects with introverted or neurotic personality types were noted as less likely to be inclined towards agile. Such findings are not unique to the Agile Development process, workplaces in general tend to favour those who are extroverted and open. As Susan Cain explains, \textit{`studies show that we rank fast and frequent talkers as more competent, likeable, and even smarter than slow ones' }\cite{Cain}. Such views are also seen in educational settings \textit{`research suggests that the majority of our teachers believe that the 'ideal student' is an extrovert.'} \cite{Cain}. This paper sets out to explore how introverted personality types can best adapt to Agile and how they can be aided in doing so.


\section{Introversion, the Games Industry and Agile}

It is often assumed that introverts are well suited to computing, \textit{`People in and out of the computing field often believe that a) technology is a field that is very attractive to introverts, and b) technology is a field that has traditionally been very amenable to introverts'} \cite{Kacz}, and certainly in early hacker culture, there is evidence to support such beliefs. Yet, in computing, especially within the games industry, teamwork has become increasingly important. Today, as opposed to the height of hacker culture, programming is conducted in teams, often large and evolving, and as such programming may not be as ideal to the introverted as it once was. 

Indeed, introverted traits often contradict with the agile manifesto: they tend to enjoy working with few interruptions, prefer not to show, or discuss work before it is completed and feel drained after only small amounts of social activity\cite{Cain}.  These traits can make working within a team difficult, and within Agile, learning to manage such traits and preferences is essential.

Whilst it could be argued that individuals who struggle with teamwork, are not suited to the game development industry, a multiplicity of personalities within a team has been found to be significantly beneficial,\textit{`The degree of homogeneity of personality of members of the groups used in this study was seen to have a direct bearing on the effectiveness of the groups in producing solutions to problems'}\cite{Hoffman}. A diverse range of personality types can enhance team performance and therefore, the whole team may benefit from supporting colleagues in adjusting to agile. Perhaps, it is important to remember that in adopting the Agile Manifesto, a team does not necessarily need to follow it as verbatim. Instead Agile can be adapted to teams by learning from both positive and negative experiences. 

\section{Potential Challenges}

To adapt to agile effectively, introverts may benefit from identifying how challenges faced in agile are influenced by their personality traits. This paper will focus on five key areas in which an introverted individual may be reasonably expected to struggle. 
\newline
\newline
1.	\textbf{Accountability} \newline
In agile, accountability ensures teams members are realistic with their workloads, contribute to the project, and helps increase trust within teams,\textit{`The accountability and collective responsibility in agile methods nurture trust by facilitating vigilance, aligning members' perceptions realistically with individual competences and abilities'} \cite{McHughConboyLang}. Introverts may feel more pressure and anxiety in terms of accountability than their more open and extroverted colleagues. Often, such pressures do not stem from other team members, instead the level of accountability involved in agile processes can increase an individuals perceived sense of responsibility \textit{`This pressure was entirely self-inflicted, but it might be consequent to the increased visibility of tasks and personal accountability to the team.'}\cite{McHughConboyLang}. Individuals who have not developed techniques to manage such pressures may use negative coping strategies, perhaps overworking in, or alternatively withdrawing from, the project.

2.	\textbf{Communication} \newline
In Agile, regular communication is encouraged through daily stand-ups, reviews, and retrospectives.  These methods are designed to increase accountability and trust, identify impediments to goals and support the production of working software. Introverted types, whom are easily drained by social activity and feel protective over work in progress, are likely to struggle with such communication methods. For some, face-to-face communication may be more challenging than using an online platform, however, research suggests this is not the most effective approach

3.	\textbf{Social Demands} \newline
A key characteristic of introverts, is their need to have adequate time alone, often feeling drained by social activities. In the team, focused environment of Agile Game Development, this can be problematic. Full immersion in teamwork and teambuilding activities has been found to increase trust and a feeling of togetherness \textit{`Constant immersion and engagement with the team as a whole, for example, and the development of rituals surrounding team activity, were seen to support the development and prevalence of a shared identity.'} \cite{WhitworthBiddle}. A natural coping strategy, may be to avoid valuable team-bonding opportunities, leading to isolation from other team members and creating friction. Introverted individuals need to find a balance between the alone time they need and their interactions within the team.

4.	\textbf{Conflicts in Personality Traits} \newline
Teams have been found to benefit from diversity in personality, a strong bias towards extraversion or introversion in a team can reduce productivity \cite{Hoffman}. However, with diversity, misunderstandings of other traits and mindsets can upset team relations "Interaction between different personalities without understanding and managing their differences can be a source of conflict." \cite{LicorishPhilpottMacDonel}, it may be helpful to educate employees in the positive and negative impacts of personality traits. Giving team members a better understanding of each other's traits, can help prevent misguided judgements of their peers. "We also found that several project management tools did not consider such risks, tending instead to focus on the handling of projects' technical challenges." \cite{LicorishPhilpottMacDonel}. Introverted types are often considered dull and self-centred, whilst extraverts are judged as superficial and insincere. Such assumptions can be damaging and can derail progress, therefore it may be valuable for Project Managers to consider personality factors when leading a team.


5.	\textbf{Recognising the Positives} \newline
The games development industry may be more generally suited to extroverted types, yet, it is important for those more introverted, to recognise how their traits can also positively contribute to a team. Research has suggested that introverted individuals are often the most creative people in a team, \cite{feist}, potentially due to their tendency towards a rich interior world and imagination. Whilst, an introvert's tendency towards reflection rather than action, can be a block to progress, it can also be valuable. Introverts tend to have greater listening skills, which can make them effective Project Managers and their tendency towards reflection before speech can lead to a better understanding of team member's capabilities. These traits have been found to make introverted individuals more effective leaders than their extroverted colleagues \cite{Morrish}, yet extroverts are commonly believed to be more capable leaders \cite{Cain}. Increasing colleagues understanding of different personality types may be helpful in better selection of roles in a team.

\section{Recommendations}

Introverted individuals may benefit from the following recommendations in adapting to agile:

1. \textbf{Take time out to recharge.}

Taking short, regular breaks during the day to prevent feeling drained by the socially demanding aspects of agile.

2. \textbf{Recognise value and participate in team-bonding opportunities.}
 
Introverted types tend to see extra social activities as an unnecessary stress, this can lead to friction and isolation within a team \cite{WhitworthBiddle}. Ensuring involvement in team-bonding opportunities may help an introverted individual better adapt to teamwork.

3. \textbf{A balanced team of extraverted and introverted traits.}

Project managers could use personality assessment tools, to ensure a balanced team of both extraverts and introverts, \textit{`The results show that placing people together in a way that balances the number of leaders and covers all necessary work roles in the group (as foreseen by the personality assessment tool) significantly improves the end result, the quality of communication, and the workers’ perceptions of being accepted and producing good results, compared to teams that have a surplus of leader types.'} \cite{Lykourentzou}. 

4. \textbf{Improve understanding of personality types in a team.}

Within balanced teams, members may benefit from increased understanding of personality types, helping to prevent ill-informed judgements of their colleagues, whether extraverted or introverted. 

5. \textbf{Recognise and utilise introverted traits that can contribute to team performance and leadership.}

Due to society's preference for extraverts \cite{Cain}, introverted qualities can be undervalued. Yet, studies have found introverts to often be more effective leaders \cite{Morrish} and more creative \cite{feist}. Recognition of such values, may help colleagues to be accepting of the introverted traits which contradict with agile processes.

6. \textbf{Managing anxiety and pressure.}

Whilst the increased accountability in agile, has been found to benefit productivity within teams, it has also been found to heighten feelings of pressure and anxiety in introverted individuals. 

\section{Conclusion}

Write your conclusion here. The conclusion should do more than summarise the essay, making clear the contribution of the work and highlighting key points, limitations, and outstanding questions. It should not introduce any new content or information.


\bibliographystyle{ieeetran}
\bibliography{references}

\end{document}