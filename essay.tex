% Please do not change the document class
\documentclass{scrartcl}

% Please do not change these packages
\usepackage[hidelinks]{hyperref}
\usepackage[none]{hyphenat}
\usepackage{setspace}
\doublespace

% You may add additional packages here
\usepackage{amsmath}

% Please include a clear, concise, and descriptive title
\title{Can the way that user stories are prioritised and allocated I.E. "MoSCoW method" have an impact on the length of a sprint in the video games industry?}

% Please do not change the subtitle
\subtitle{COMP150 - Agile Development Practice}

% Please put your student number in the author field
\author{0481929291}

\begin{document}

\maketitle

\abstract{My agile development question revolves around one specific part of the Agile Development process and that is "User Stories", and how I intend to answer the aforementioned question is to delve deeper into first, what a User Story is and how it is used by members of the development team in the development cycle to create a finished product, and then go further and look at how the user stories are created in the first place, and thus prioritised amongst the team members and on the Scrum Board, and as there are different methods of prioritising these user stories such as Walking Skeleton and the MoSCow method, I will compare and contrast these individual methods, and by using my research gained from my noted sources, I will be able to ascertain which is the most efficent for teams in a games development environment. And from these reflections on the different methods and the ways that they change the development teams behaviour and work pattern I will be able to draw a final conclusion on my question and wheter futher research is required to gain a better understanding of the impact it can have on the development time of a project in the video games industry.}

\section{Introduction}

Write your introduction here. A brief introduction is recommended, which should outline key details of the chosen topic and the reviewed papers, motivate the work, and provide a roadmap of key points to the reader. The motivation is quite important here, as essays should have a contribution (i.e., what is the point of the essay, and what does the reader take away from the essay) and the link between the motivation (in the introduction) and the contribution (in the conclusion) should be made clear.

\section{Your section title here}

Write the main body of your essay here. Add more sections if appropriate. You may choose to write about each of your three papers in its own section, or you may choose a different structure. Either way, remember that you are being assessed on technical insight and analysis: it is not enough to merely summarise the contents of the three papers. You must demonstrate the ability to make inferences beyond what is written in the papers, and to draw the three papers together into a single coherent narrative.

Your essay must make a clear recommendation, in terms of which of the three techniques you have reviewed is the best according to whichever metric or metrics you feel is most appropriate. You must justify your choice, backing it up with empirical evidence. However remember that an academic essay is not a murder mystery: you should already have briefly discussed your recommendation in the introduction and in other parts of the essay. Do not save it for a grand reveal at the end.

\section{Conclusion}

Write your conclusion here. The conclusion should do more than summarise the essay, making clear the contribution of the work and highlighting key points, limitations, and outstanding questions. It should not introduce any new content or information.

\bibliographystyle{ieeetran}
\bibliography{references}

\end{document}
