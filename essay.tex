% Please do not change the document class
\documentclass{scrartcl}

% Please do not change these packages
\usepackage[hidelinks]{hyperref}
\usepackage[none]{hyphenat}
\usepackage{setspace}
\doublespace

% You may add additional packages here
\usepackage{amsmath}

% Please include a clear, concise, and descriptive title
\title{How does running daily stand-ups in a virtual office setting, using tools such as Slack, influence time-boxing and team productivity?}

% Please do not change the subtitle
\subtitle{COMP150 - Agile Development Practice}

% Please put your student number in the author field
\author{1607934}

\begin{document}

\maketitle

\abstract{In the vast field of computer science including software development, computer networks, and many other computing areas, you will always encounter barriers and time constraints whether you expect them or not. Suddenly your colleague is absent one day, forcing you to improvise and finish up the rest of your group work before the approaching deadline, or maybe you are having an internal problem that is considerably prohibiting team productivity. For these reasons, I will be answering the following research question: How does running daily stand-ups in a virtual office setting, using tools such as Slack, influence time-boxing and team productivity?}

\section{Introduction}

I will be focusing on how virtual tools influence time-boxing & team productivity, the importance of using virtual tools and the various effects they evoke on different people. I have chosen to explore this question because I feel like I have enough personal experience to discuss about this certain topic, something that I think will reinforce my facts and opinions. Additionally, I believe it to be an especially important issue in today's world of business and technology, where we can merge both to provide and receive good and realistically timed products.


\section{Your section title here}

Firstly, you could argue that virtual tools are not essential. They are not required at all if, say, a team is fine at their current state of productivity. Why change what's not broken? Although, while it may temporarily tamper with progress, learning to cope and adapt to using virtual tools can vastly improve time-boxing, which thereafter positively affects team productivity. Even if it is minimal, simply communicating for a maximum of five minutes - which is more commonly known as as a daily stand-up - with your colleagues about your tasks and barriers is enough to provide information to the rest of the group, which they in turn could lend some assistance in if needed. This has an excellent effect in the long run, as it develops a sense of positivity and reassurance amongst group members, resulting in elevated levels of team productivity, which consecutively decreases the potential problems that could arise in terms of time-boxing.

\section{Conclusion}

Write your conclusion here. The conclusion should do more than summarise the essay, making clear the contribution of the work and highlighting key points, limitations, and outstanding questions. It should not introduce any new content or information.

\bibliographystyle{ieeetran}
\bibliography{references}

\end{document}
