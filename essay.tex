% Please do not change the document class
\documentclass{scrartcl}

% Please do not change these packages
\usepackage[hidelinks]{hyperref}
\usepackage[none]{hyphenat}
\usepackage{setspace}
\doublespace

% You may add additional packages here
\usepackage{amsmath}

% Please include a clear, concise, and descriptive title
\title{What are the key challenges game developers encounter when applying agile in a remote context whilst using SCRUM?}

% Please do not change the subtitle
\subtitle{COMP150 - Agile Development Practice}

% Please put your student number in the author field
\author{1605240}

\begin{document}

\maketitle

\abstract{The use of agile is becoming more popular with teams of all sizes within the games industry, however when distance they face challenges. This essay will be looking into the key challenges game developers face whilst applying agile in a remote context and searching for ways to make this process as efficient as possible. In order to accomplish my task, I will be sourcing papers from academics and authors on what problems they have faced whilst using agile in a remote context and also papers which talk about solutions and I will discuss my views on whether or not I think they are suitable.}

\section{Introduction}

This essay will be reviewing the adoption of the agile methodology and the key challenges game developers face when it is being used in a remote context with distributed teams. The agile methodology is a method that focuses on the quality of software produced not the amount and how they can make the software better. Most teams of game developers use SCRUM which is an agile development method. SCRUM is an iterative approach where teams do two to four week sprints where they prioritise certain user stories they have in a product backlog. They hold daily stand ups where every team member answers three questions, what did you do yesterday? What are you going to do today? And what problems did you encounter/encountering? To make SCRUM as effective as it can be the main thing is good communication between team members, yet when teams are distributed they encounter challenges, the main one being communication and time zones which I will be looking into and trying to suggest alternatives to them.

\section{Your section title here}

Communication is the biggest issue encountered by teams using agile in a remote context, however communication is very broad and can be split up into many different smaller issues, and the first of these issues is how do they host the scrum if everyone is split up? They use some of the many important tools they have available to help them such as web conferencing and the daily scrum meetings online so they can all discuss how they are getting on, what they are going to do, if they’ve got any problems and also it helps put a face to an online name [6]. I think this is important because it tells everyone what everyone is doing and you can offer to help them and vice versa and it’s a lot friendlier when you see someone’s face instead of speaking them online which helps build up trust and a friendship making you care more about the team. The lack of communication between team members can lead to team members not getting the same vision of the project, and not knowing the task at hand so they had to ask multiple times to understand clearly what they had to do [5], this takes up more than it should as the daily scrums should be fifteen minutes’ maximum. To fix this communication between team members should be clear and instructions should be precise so no questions are needed.
Another pressing issue that links into the last issue is when teams are based in different parts of the globe their time zones can vary making it hard for teams to sync up working times [9] or more importantly share work with each other. When sharing work with each other teams are sometimes break the code making it hard for the other team to figure out what they’ve done wrong, what most teams have implemented to fix this is they don’t leave the office until the last code that was built didn’t break the build they were working on [6]. This works well meaning no time is wasted for the other team having to try and fix the code.
One other issue that can have quite an impact and difficult to avoid is language barriers, because if people speak a different language [11] teams have to get tools to translate and makes communication more of an issue. Part of this issue is also when people have different cultures as the quality of the work can be harder to manage as expectations can sometimes be “culturally influenced” [8] this can be a problem that can be solved with the daily scrum meetings being online so every team member can get a vision of the game and know the quality it should be.
As mentioned above sharing work is a big part of how smoothly the project runs, the teams should set up shared repository [6] so they can all access the work and help each other if people can’t manage to do a task, teams also have programs that allow multiple people editing the code at once, this benefits the team massively as it saves the team having to send around the latest versions of code to each other, it just gets synced with everyone’s computer. 


\section{Conclusion}

Write your conclusion here. The conclusion should do more than summarise the essay, making clear the contribution of the work and highlighting key points, limitations, and outstanding questions. It should not introduce any new content or information.

\bibliographystyle{ieeetran}
\bibliography{references}

\end{document}
