% Please do not change the document class
\documentclass{scrartcl}

% Please do not change these packages
\usepackage[hidelinks]{hyperref}
\usepackage[none]{hyphenat}
\usepackage{setspace}
\doublespace

% You may add additional packages here
\usepackage{amsmath}

% Please include a clear, concise, and descriptive title
\title{How does running daily stand-ups in a virtual office setting, using tools such as Slack, influence time-boxing and team productivity?}

% Please do not change the subtitle
\subtitle{COMP150 - Agile Development Practice}

% Please put your student number in the author field
\author{1607934}

\begin{document}

\maketitle

\abstract{In the vast field of computer science including software development, computer networks, and many other computing areas, encountering barriers and time constraints is guaranteed to happen whether they are expected or not. Suddenly a colleague is absent one day, forcing improvisation among team members in order to finish the rest of the group work before the approaching deadline, or maybe an internal problem is considerably prohibiting team productivity. For these reasons, the following research question that will be addressed is this: How does running daily stand-ups in a virtual office setting, using tools such as Slack, influence time-boxing and team productivity?}

\section{Introduction}

In conjunction with the research question, the various topics that will be tackled are as follow: how virtual tools influence time-boxing and team productivity, the importance of using virtual tools and the various effects they evoke on different personality traits. Furthermore, agile processes in the game development industry that emphasise self-organised teamwork and communication will be also addressed \cite{Manifesto}. These have been chosen to be explored as a result of personal experience and evidence from the referenced papers. Additionally, it is an especially important issue in today's world of business and technology, where we can merge both to provide and receive high quality and realistically timed products.


\section{Influence of virtual tools and Agile}

It is important to note that agile processes rely on feedback and team communication to work \cite{Manifesto, Steve}. An example of such process is called a "daily stand-up". In short, they are time-boxed meetings ranging from five to fifteen minutes between team members to discuss about their current work process, all this while standing up in order to keep everyone awake and engaged - hence the name. \cite{McHugh} Transferring this to a virtual setting such as Slack, however, changes it all considerably. The fact that doing it virtually diminishes the act of standing, as well as communicating face to face which contradicts the whole system \cite{Lehmann}. Although, on the other hand, they have notable benefits for, say, remote teams. Communication becomes possible regardless of a team members' location or time zone, reinforcing productivity while wasting as little time and money as possible. Of course, it also has its own set of disadvantages, such as the dependency on the group members - trusting each member to truly show up at the agreed time slot of the virtual meeting, which is known to be easily forgettable and should not be taken lightly as it has the potential to demotivate committed members - and the lack of social bonding between group members, which might occur after inherently averting away from face to face meetings. \cite{Lehmann} Certainly, the project tasks themselves are achievable this way, however the camaraderie and unification developed in live meetings remain unmatched.

\section{Importance of virtual tools and the games industry}
The use of agile methodologies in the game development industry has become quite common \cite{Andre}. However, when adding in virtual tools in the mix, one could argue that they, the virtual tools themselves, are not essential. They are not required at all if, say, a team is fine at their current state of productivity. Why change what is not broken? Although, while it may temporarily tamper with progress, learning to cope and adapt to using virtual tools along with the Agile methodology can vastly improve time-boxing which thereafter positively affects team productivity, which is why it is an attractive approach in the game development industry. Even if it is minimal, simply communicating for a maximum of five minutes with colleagues about current tasks and barriers is enough to provide information to the rest of the group, which they in turn could lend some assistance in if needed. This has a positive effect in the long run, as it develops a sense of positivity and reassurance amongst group members, resulting in elevated levels of team productivity, which consecutively decreases the potential problems that could arise in terms of time-boxing such as not reaching the specified date.

\section{Different personality traits}
Personality traits have been attractive aspects of discussions in the computing field, the two notable ones being introvertism and extrovertism. The Agile approach is known to involve continuous communication between teams which could potentially take up much of the day, something that ultimately might not seem as attractive among introverts. Indeed, the Agile methodology has brought upon a much-needed touch both in computing and business areas where products can be delivered in an interactive and flexible manner - in contrast to the waterfall methodology -, however there is no doubt that it favours extroverted personalities. Being able to engage in quick yet elaborate meetings, demonstrate incomplete work, tolerate critique and generally unmask one's self is crucial when adapting to the Agile methodology. Of course, this is harmless and tolerable for a small amount of time, although the problem gradually intensifies when having to sustain the particular "act", which as expected can lead to exhaustion and eventually decrease team productivity. With all this considered, virtual tools come as an advantageous touch for introverts. No longer is the need to attend live meetings or expose one's self necessary, not when asynchronous communication is possible. This will ensure comfort and time to be able to fully form their own points and opinions in contrast to meetings in real-time, which may create overwhelming and draining emotions. 

\section{Conclusion}
- Currently in motion -

\bibliographystyle{ieeetran}
\bibliography{references}

\end{document}
