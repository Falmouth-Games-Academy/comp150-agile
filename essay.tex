% Please do not change the document class
\documentclass{scrartcl}

% Please do not change these packages
\usepackage[hidelinks]{hyperref}
\usepackage[none]{hyphenat}
\usepackage{setspace}
\doublespace

% You may add additional packages here
\usepackage{amsmath}

% Please include a clear, concise, and descriptive title
\title{What are the key benefits and drawbacks of daily scrum meetings and how could they be leveraged or overcome in a game development project?}

% Please do not change the subtitle
\subtitle{COMP150 - Agile Development Practice}

% Please put your student number in the author field
\author{1603748}

\begin{document}

\maketitle

\abstract{Please include an abstract of at most 100 words (these do not count towards your word count).}

\section{Introduction}

In this essay I will outline the key benefits and drawbacks of daily scrum meetings and using my knowledge gained from research to discuss how they could be leveraged or overcome to make daily scrum meetings as effective as possible. I will first outline what a daily scrum meeting is and why it is important and useful, I will then discuss the benefits, drawbacks, and solutions to those drawbacks of efficient daily scrum meetings found in my research and give my own conclusions, followed by the final conclusion. In this paper I hope to provide solutions to the most common obstacles to having an efficient daily meeting and highlight good practice in a daily meetings. After reading this you should know how to make the most out of daily scrum meetings and be as productive as possible.

\section{Daily Scrum}

As the agile manifesto \cite{Manifesto} states ``The most efficient and effective method of 
conveying information to and within a development 
team is face-to-face conversation''. A daily scrum meeting is a daily face to face 15-minute meeting, allowing the members of a team to inspect the progress of work and to remove impediments. It is established by the Scrum master and takes place each day at the same place and time, thus avoiding unnecessary coordination. Each team member answers the following three questions: (1) What have you worked on since the last meeting? (2) What will you work on until the next meeting? (3) Have any impediments occurred? \cite{ScrumEveryDay}. It is the Scrum masters responsibility to resolve any impediments that arise during the meeting. Team members meet in front of a task board and update their status of work \cite{Multitouch}, allowing team to check the status of sprint backlog items on a daily basis and thus keeps the team focused \cite{ScrumEveryDay}.

\section{Obstacles to Efficient Meetings}

One of the keys to an efficient daily meeting is to complete the meeting at the right time of day for the right amount of time in order to be productive by not wasting time. In a case study I researched \cite{Obstacles} it was found that there were many temporal obstacles to an efficient meeting. They observed that people often arrived late to meetings which meant the meeting would not start on time. This means that time is wasted for those who arrive on time. It was also found that many of the meetings in the case study lasted longer than 15 minutes. This was usually caused by people doing things like doodling on their notepad or using their phones. It was found that meetings lasted between 7 and 45 minutes with a mean of 22 minutes, which meant that the cost in time of the meeting was 47 percent more than planned. In addition to these obstacles, the meetings were not always at the same time of day. In cases where people have their daily meeting at the start of each day this is an issue. This is because team members wouldn't usually start tasks from the backlog until after the daily meeting, so if the daily meeting doesn't always happen at the start of day a lot of time will be wasted. Along with all of these temporal obstacles, research suggests that many team members perceived the meeting as an interruption of their work flow and would prefer fewer meetings. One paper discussed how having meetings every day can be bad in some cases. For example in cases where employees are assigned to several teams at the same time, daily meetings in each of the teams would lead to severe constraints on behalf of those employees \cite{ScrumEveryDay}.

\section{Solutions}

As you can see there are many common obstacles related to time preventing daily scrum meetings from being as productive as possible. I found a number of possible solutions to each of the obstacles. To make sure that meetings start on time, and at the same time every day, an alarm could be used to signal the start of the meeting \cite{Obstacles}.  A timer could be used to make sure that the meeting lasts 15 minutes long, it should be visible to every member in the meeting so they know how long is left. Many team members perceived the meetings as an interruption to their work flow, this was most likely due to how long the meetings were taking (up to 45 minutes in some cases), so if the previous solutions were implemented this might not be a problem anymore. This would mean the daily scrum meeting would still be used as intended, on a daily basis.    

\section{Benefits of daily meetings}

I plan to talk about the benefits of daily scrums in this section and discuss how daily scrums can be leveraged. I will reference \cite{Distributed,SupportingScrum,SoftwareProjects} 

\section{Conclusion}

Write your conclusion here. The conclusion should do more than summarise the essay, making clear the contribution of the work and highlighting key points, limitations, and outstanding questions. It should not introduce any new content or information.

\bibliographystyle{IEEEtran}
\bibliography{references.bib}


\end{document}
