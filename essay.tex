% Please do not change the document class
\documentclass{scrartcl}

% Please do not change these packages
\usepackage[hidelinks]{hyperref}
\usepackage[none]{hyphenat}
\usepackage{setspace}
\doublespace

% You may add additional packages here
\usepackage{amsmath}

% Please include a clear, concise, and descriptive title
\title{Should the technical team have a separate scrum meeting to the rest of the development team?}

% Please do not change the subtitle
\subtitle{COMP150 - Agile Development Practice}

% Please put your student number in the author field
\author{1604629}

\begin{document}

\maketitle

\abstract{Within this essay I thoroughly discuss the concept of having not just one but multiple scrum meetings depending on the scale of the team, or types of team. An example of this would be a team may hold a scrum meeting to plan their day out and then the technical team may have a further meeting to go more in depth on how they plan on achieving these goals. I have read several other papers that discuss what scrum is, how it works and if teams have been able to effectively hold more than one meeting and reference them all here.}

\clearpage

\section{Introduction}

In the world of software development, the agile methodology has become the standard methodology. Scrum is the most commonly used process within Agile, which focuses on delivering functioning software regularly as opposed to just one deliverable when a complete model has been made. Based upon this information we can see that investor involvement is pretty much essential as they will be giving regular feedback on each deliverable. However, should this be the only input any external party should have? Should investors be allowed to sit in during team meetings or stand ups for example? How can investors provide input safely without interfering with the team flow? With this in mind if investors were able to take part in meetings/stand-ups could this negatively impact the teams performance? Should communications with the investors be done by only the scrum master who could then relay it back to the team, therefore keeping them out of team meets?

\section{What is Agile?}
Agile is one of the most popular methodologies within the software development industry, with scrums being a major aspect of the process. [1] The Agile methodology focuses more on customer collaboration over negotiation, responding to change over following a plan and regular working software over comprehensive documentation. Most notably, agile focuses on iterations or sprints which take places over a period of time with a working deliverable at the end of each sprint. The sprint is built on the basis of being an iterative development process, after each sprint you would have a functioning application which you can show to the client and receive feedback and ideas which you can work on in the following sprint.

\section{How do Scrums meetings work?}
The scrum meetings area a huge part of this process, taking place usually every morning for ideally around 15 minutes. These are also known as stand up meetings, each team member of the team would in turn talk about what they did the previous day and what they plan on achieving this day [2]. By holding the meetings standing up and limiting them to 15 minutes I see how this keeps the discussion short and concise, making sure people only mention the important things. Something I found interesting is that some teams offer a mark members as committed to the work and only those committed to the work may speak in the meetings. This would also keep the conversations on topic and useful, as people who aren't very committed to the project are obviously going to find the meetings uninteresting or boring and may cause a negative impact on the team [3].
With this in mind my argument would be that due to the meetings only being 15 minutes each morning, surely the technical members of the team could comfortably hold a second meeting to go more in to detail of how they plan to achieve the goals set out in the first, without taking too much time away from the work day. 

\section{Could a technical team benefit from having a separate meeting?}
Write the main body of your essay here. Add more sections if appropriate. You may choose to write about each of your three papers in its own section, or you may choose a different structure. Either way, remember that you are being assessed on technical insight and analysis: it is not enough to merely summaries the contents of the three papers. You must demonstrate the ability to make inferences beyond what is written in the papers, and to draw the three papers together into a single coherent narrative.

Your essay must make a clear recommendation, in terms of which of the three techniques you have reviewed is the best according to whichever metric or metrics you feel is most appropriate. You must justify your choice, backing it up with empirical evidence. However remember that an academic essay is not a murder mystery: you should already have briefly discussed your recommendation in the introduction and in other parts of the essay. Do not save it for a grand reveal at the end.

\section{Conclusion}

Write your conclusion here. The conclusion should do more than summaries the essay, making clear the contribution of the work and highlighting key points, limitations, and outstanding questions. It should not introduce any new content or information.

\bibliographystyle{ieeetran}
\bibliography{references}

\end{document}