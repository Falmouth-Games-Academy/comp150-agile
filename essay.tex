% Please do not change the document class
\documentclass{scrartcl}

% Please do not change these packages
\usepackage[hidelinks]{hyperref}
\usepackage[none]{hyphenat}
\usepackage{setspace}
\doublespace

% You may add additional packages here
\usepackage{amsmath}

% Please include a clear, concise, and descriptive title
\title{How long should a sprint last\\when developing games?}

% Please do not change the subtitle
\subtitle{COMP150 - Agile Development Practice}

% Please put your student number in the author field
\author{1703086}

\begin{document}

\maketitle

\abstract{Please include an abstract of at most 100 words (these do not count towards your word count)}

\section{Introduction}
\iffalse
Write your introduction here. A brief introduction is recommended, which should outline key details of the chosen topic and the reviewed papers, motivate the work, and provide a roadmap of key points to the reader. The motivation is quite important here, as essays should have a contribution (i.e., what is the point of the essay, and what does the reader take away from the essay) and the link between the motivation (in the introduction) and the contribution (in the conclusion) should be made clear.
\fi
This Essay will be about Scrum methodology and the importance of the work cycles that are practised within Scrum, known as Sprints, and more importantly how long a sprint should last.\\
It will also be reviewing multiple different sources that discuss varying opinions on the length of sprints and what the ideal length should be depending on the scale of the project.\\
It will start of by explaining the importance of Sprints in Scrum for games development and then go into why Sprints can be of different lengths to then argue on whether there is an ideal length for sprints to last, it will further go on to arguing that 1 week long sprints are probably the most ideal time lengths for small teams, and it will finally end with a conclusion of the argument and essay.


\section{So let's talk about sprints}
\iffalse
Write the main body of your essay here. Add more sections if appropriate. You may choose to write about each of your three papers in its own section, or you may choose a different structure. Either way, remember that you are being assessed on technical insight and analysis: it is not enough to merely summarise the contents of the three papers. You must demonstrate the ability to make inferences beyond what is written in the papers, and to draw the three papers together into a single coherent narrative.

Your essay must make a clear recommendation, in terms of which of the three techniques you have reviewed is the best according to whichever metric or metrics you feel is most appropriate. You must justify your choice, backing it up with empirical evidence. However remember that an academic essay is not a murder mystery: you should already have briefly discussed your recommendation in the introduction and in other parts of the essay. Do not save it for a grand reveal at the end.
\fi

Developing digital games is hard and it gets even harder when you need to develop games with a team, so much more can go wrong when multiple people are working together on the same project, this is why a lot of people within the games industry practice Agile development methods; such as Scrum.
\\~\\
Scrum has an incremental development process known as Sprints, Sprints are pre-planned time periods in which the team has decided on what they will work on during the sprint. Sprints are a very important part of the Scrum methodology as they give a specific time limit for the team to work on tasks to implement into the project.\\
At the end of a Sprint, a sprint review meeting occurs in which the team gets together to discuss what happened during the sprint and update their project to the latest build of the game with all the new features and fixes that have happened during the sprint.\\
Some developers like to use sprint reviews as a way of updating their fans on what they have been working on lately for the game, other developer who have early access games might even update their game after each sprint to give the fans regular new features and bug fixes in order to keep the game fresh for the players.
\\~\\
So Sprints are an important part of the development cycle for games and they need to be well planned and organized in order for the team to work efficiently on the game.

\section{So how long should a sprint last?}
Most of the sources this paper cites have said that sprints should last 2-4 weeks depending on the team and the scale of the project !!CITATION NEEDED!!\cite{two} \cite{three} \cite{four} \cite{five}. Some sources that have been explored also suggest that shorter sprints are generally better for software and games development !!CITATION NEEDED!!. Shorter sprints generally work in favour for the Agile Scrum methodology for being short iterative steps towards building a game. Any more than 4 weeks is usually considered too long.

\section{The idea of 1 week sprints?}
Some sources say that 1 week sprints are too short and don't give enough time for teams to work everything they wanted to work on from their sprint planning !!CITATION NEEDED!!, other sources say that 1 week sprints aren't good because it means a loss of time from having to do more sprint planning meetings and review meetings. This essay disagrees with these statements and will propose that 1 week long sprints are actually a very good time period for sprints,
\\~\\
Arguments for 1 week sprints: quicker/faster work flow, more iteration, more gets implemented in a shorter amount of time, great for small teams, allows for weekly updates for the fans.
\\~\\
Arguments against 1 week sprints: to many sprint meetings, not enough time for quality development.

\section{Conclusion}
\iffalse
Write your conclusion here. The conclusion should do more than summarise the essay, making clear the contribution of the work and highlighting key points, limitations, and outstanding questions. It should not introduce any new content or information.
\fi
In the end, the length of a sprint entirely depends on what the team decides, they get to choose what will best suite their needs for developing a game depending on the project's scale and scope.
\bibliography{references}
\bibliographystyle{ieeetran}


\end{document}
