% Please do not change the document class
\documentclass{scrartcl}

% Please do not change these packages
\usepackage[hidelinks]{hyperref}
\usepackage[none]{hyphenat}
\usepackage{setspace}
\doublespace

% You may add additional packages here
\usepackage{amsmath}

% Please include a clear, concise, and descriptive title
\title{Playtesting during agile game development, creating an enjoyable product}

% Please do not change the subtitle
\subtitle{COMP150 - Agile Development Practice}

% Please put your student number in the author field
\author{DO NOT WRITE YOUR NAME\\Put your student number (on your ID card) here}

\begin{document}

\maketitle

\abstract{This paper looks at how frequent playtesting during agile research development helps create a more enjoyable product for the end user, it does this by looking at different research documents centred around agile testing processes. There is a focus on how enjoyment can only be measured correctly with interaction between the target audience and the game. Going on to summarise that early playtesting using agile is possible for all sizes of projects and saves time by highlighting potential changes early in the development cycle. Because these changes are driven by the target audience the finished product is more enjoyable for the user.}

\section{Introduction}

Playtesting is a process of all game development, however there are multiple ways to go about it. This paper will be advocating for frequent, iterative playtesting throughout the development cycle using target audience playtesters and the agile philosophy. With an outcome of increased enjoyment for the end user and time saved in product changes and bug fixing in late stage development. The difference between target audience and professional playtesting will be looked at and how to have these combined saving additional time. Along with showing the viability for this process to be used in any size project, altogether showing the best way to move forward with playtesting in the game development industry.

\section{Your section title here}

Write the main body of your essay here. Add more sections if appropriate. You may choose to write about each of your three papers in its own section, or you may choose a different structure. Either way, remember that you are being assessed on technical insight and analysis: it is not enough to merely summarise the contents of the three papers. You must demonstrate the ability to make inferences beyond what is written in the papers, and to draw the three papers together into a single coherent narrative.

Your essay must make a clear recommendation, in terms of which of the three techniques you have reviewed is the best according to whichever metric or metrics you feel is most appropriate. You must justify your choice, backing it up with empirical evidence. However remember that an academic essay is not a murder mystery: you should already have briefly discussed your recommendation in the introduction and in other parts of the essay. Do not save it for a grand reveal at the end.

\section{Conclusion}

Write your conclusion here. The conclusion should do more than summarise the essay, making clear the contribution of the work and highlighting key points, limitations, and outstanding questions. It should not introduce any new content or information.

\bibliographystyle{ieeetran}
\bibliography{references}

\end{document}
