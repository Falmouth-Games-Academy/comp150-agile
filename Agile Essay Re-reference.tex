% Please do not change the document class
\documentclass{scrartcl}

% Please do not change these packages
\usepackage[hidelinks]{hyperref}
\usepackage[none]{hyphenat}
\usepackage{setspace}
\doublespace

% You may add additional packages here
\usepackage{amsmath}

% Please include a clear, concise, and descriptive title
\title{How does agile methodology affect the workload on a team when in practise?}

% Please do not change the subtitle
\subtitle{COMP150 - Agile Development Practice}

% Please put your student number in the author field
\author{1608289}

\begin{document}

\maketitle

\abstract{Proposal}

The topic of my proposed essay is how forms of agile methodology affect the workload on a team when in practise in a workplace. In the recent past few years many companies and also development studios have been using scrum as their form of agile methodology. Throughout this essay I will go through both the pros and cons to putting scrum into practise and how it affects the workload on employees. Through doing so I can address many affects it will have on the quality of work, efficiency of work and also morale of a team during the scrum process. 

\section{Introduction}
The purpose of this essay is to address the effects of the workload on a team when in practise within a game development company. Throughout this essay it will address the pros and cons to how forms of agile methodology can influence the behaviour of a development team. The one agile method this essay will cover is the use of scrum within a working practise. Throughout the past few years the use of agile methodology has become increasingly more popular amongst a varied form of businesses and companies. Specifically, the use of agile methodology within the game development infrastructure. This is due to the simplicity and also benefits a team can take advantage of to increase the quality of work and also the overall efficiency compared to other methods of meeting a deadline with the intended final product. As well as the benefits scrum can provide to a development team, it can also provide a few cons too. 

\section{Trello board}
"The Development Team usually starts by designing the system and the work needed to convert the Product Backlog into a working product Increment. Work may be of varying size, or estimated effort. However, enough work is planned during Sprint Planning for the Development Team to forecast what it believes it can do in the upcoming Sprint." \cite{scrum alliance}

Before the sprint the team will contribute towards a board of tasks which will be designed to have multiple stages. Throughout these stages each task can vary in size and also some tasks will be prioritised over others depending on how intense each task is. By eliminating these tasks first, although the work load will be the same, it will allow the team to evaluate the task in review which may then be taken back a stage for another member to adapt to complete to the highest standard. By doing so it enables the reprioritised task which may be more valuable to the project won’t be rushed and can be completed to a near perfect standard while also allowing other tasks to be completed on time.

\section{Development skills}
"Scrum emphasizes empirical feedback, team self-management, and striving to build properly tested product increments within short iterations." \cite{agile methodology rss}

Scrum can provide the team with multiple skills that will assist them later on with future projects and endeavours. The team will learn to review completed tasks that have been achieved by their peers, by doing so they can give a critical analysis of the work completed and improve upon it where needed. The team will also self-manage themselves when the scrum is in practise. Throughout the team will complete various pieces of work individually, at their own pace but should be keeping up to date with the sprint deadline set by the scrum master. The team will also push themselves to complete work as efficient as possible but up to the highest standard of which they can achieve to work within the brief, rather than assigning each team member a specific role they will complete work by one piece at a time and then move onto the next piece of work. By doing so, this spreads out the workload of the sprint amongst the team and also therefore reduces the stress levels amongst the team.

\section{Cons}
Although there are many benefits to scrum, there are also many cons to it too. 
"The Cons
•	Scrum often leads to scope creep, due to the lack of a definite end-date
•	The chances of project failure are high if individuals aren't very committed or cooperative
•	Adopting the Scrum framework in large teams is challenging
•	The framework can be successful only with experienced team members
•	Daily meetings sometimes frustrate team members
•	If any team member leaves the project in the middle, it has a huge negative impact on the project
•	Quality is hard to implement, until the team goes through aggressive testing process" \cite{2017}

Although the abstract above outlines most, if not all the limitations to scrum and all have an effect on the workload of the scrum, a few of these limitations can be avoided if the method is practised appropriately.

Project failure is one of the largest cons while using scrum as a method of agile. If one or many of the members of a team do not appropriately cooperate or commit themselves to the sprint in-hand then it will penalise the whole team. As a result of this, the whole team may be under the illusion that the tasks which have been completed are to a standard of which is acceptable to fit the intended brief, this can entail the project to be a failure once each sprint has been completed before the official deadline of the final product. A team member may not be committed due to the task they have undertaken; this may be because the workload of the task may be considerably larger than any of the other tasks.
 
A team member leaving the project mid-way through can prove to be another major con that can occur during the duration of the project. This can have a severe negative impact on the whole team, this is due to the increase of the workload which has to be distributed between the rest of the team. This can make work be rushed and below a suitable standard which could potentially affect the project so negatively that it could be scrapped completely. Ensuring each team member is committed to the project is an essential part of any project. 

\section{Conclusion}
To answer the question of, ‘how does agile methodology affect the workload on a team when in practise?’ is as follows: Agile methodology has many factors that need to be taken into account when considering practising the method of development. The final product of a development team that make use of scrum to produce a product can be immensely worthwhile and also very rewarding, but it can also be prove to scrap the project completely if not been put into practise appropriately with great amounts of consideration. 

Write your conclusion here. The conclusion should do more than summarise the essay, making clear the contribution of the work and highlighting key points, limitations, and outstanding questions. It should not introduce any new content or information.

\bibliographystyle{ieeetran}
\bibliography{references}

\end{document}