\documentclass{scrartcl}

\usepackage[hidelinks]{hyperref}
\usepackage[none]{hyphenat}

\title{Essay Proposal}
\subtitle{COMP150 - Agile Essay}

\author{Madeleine Kay}

\begin{document}

\maketitle

\section*{Topic}
In this essay I intend to address the question are Agile measurement methods such as burn down charts suitable for use in games development? I will look at different Agile metrics and their applications.....

\section*{Paper 1}
\begin{description}
	\item[Title:] Designing and implementing a measurement program for Scrum teams: what do agile developers really need and want?
	\item[Citation:] \cite{Ktata}
	\item[Abstract:] ''Agile developers are generally reluctant to non-agile practices. Promoted by senior software practitioners, agile methods were intended to avoid traditional engineering practices and rather focus on delivering working software as quickly as possible. Thus, the unique measure in Scrum, a well known framework for managing agile projects, is velocity. Its main purpose is to demonstrate the progress in delivering working software. In software engineering (SE), measurement programs have more in depth purposes and allow teams and individuals to improve their development process along with providing better product quality and control over the project. This paper will describe the experience and the approach used in an agile SE company to design and initiate a measurement program taking into account the specificities of their agile environment, principles and values. The lessons learned after five months of investigation are twofold. The first one shows how agile teams, in comparison to traditional teams, have different needs when trying to establish a measurement program. The second confirms that agile teams, as many other groups of workers, are reluctant and resistant to change. Finally, the preliminary results show that agile people are more interested in value delivery, technical debt, and multiple aspects related to team dynamics and will cooperate to the collection of data as soon as there tools can do it for them. It is believed that this research could suggest new guidelines for elaborating specific measurement programs in other agile environments.''
	\item[Web link:] \url{http://dl.acm.org/citation.cfm?id=1822341}
	\item[Full text link:] \url{https://www.researchgate.net/profile/Ghislain_Levesque/publication/221186048_Designing_and_implementing_a_measurement_program_for_Scrum_teams_what_do_agile_developers_really_need_and_want/links/0fcfd50574f7580b5c000000.pdf}
	\item[Comments:] This paper talks about the problems with trying to measure or predict the progress of an Agile project. It mentions the problems with user stories being reopened and adding more which could be linked to the games industry. 
\end{description}

\section*{Paper 2}
\begin{description}
	\item[Title:] Why are industrial agile teams using metrics and how do they use them?
	\item[Citation:] \cite{Kupiainen}
	\item[Abstract:] '' Agile development methods are increasing in popularity, yet there are limited studies on the reasons and use of metrics in industrial agile development. This paper presents preliminary results from a systematic literature review. Based on our study, metrics and their use are focused to the following areas: Iteration planning, Iteration tracking, Motivating and improving, Identifying process problems, Pre-release quality, Post-release quality and Changes in processes or tools. The findings are mapped against agile principles and it seems that the use of metrics supports the principles with some deviations. Surprisingly, we find little evidence of the use of code metrics. Also, we note that there is a lot of evidence on the use of planning and tracking metrics. Finally, the use of metrics to motivate and enforce process improvements as well as applicable quality metrics can be interesting future research topics. ''
	\item[Web link:] \url{http://dl.acm.org/citation.cfm?id=2593873}
	\item[Full text link:] \url{https://www.researchgate.net/profile/Mika_Maentylae/publication/266657618_Why_are_industrial_agile_teams_using_metrics_and_how_do_they_use_them/links/54734bea0cf2d67fc0361916.pdf}
	\item[Comments:] This paper reviews a series of paper about different Agile metrics and talks about the most common ones, the reference section includes lots of papers that may be useful. 
\end{description}

\section*{Paper 3}
\begin{description}
	\item[Title:] Survey on agile metrics and their inter-relationship with other traditional development metrics
	\item[Citation:] \cite{Misra}
	\item[Abstract:] ''In our civilized world today, measurement is very important in every aspect of our lives as a means of quantifying our success or progress in whatever activity we involve ourselves in. Consequently, this paper outlines the various metrics that are utilized in the Agile development process and compares them with the ones used in time past to measure success and progress. ''
	\item[Web link:] \url{http://dl.acm.org/citation.cfm?id=2047430}
	\item[Full text link:] \url{https://www.researchgate.net/profile/Sanjay_Misra2/publication/220631207_Survey_on_agile_metrics_and_their_inter-relationship_with_other_traditional_development_metrics/links/5578816a08aeacff20028500.pdf}
	\item[Comments:] This paper looks traditional and Agile metrics, it talks about how many possible metrics there are how they can vary depending on the software being produced. This should be relevant as the Agile metrics used in game making would likely be different than a piece of commercial software.
\end{description}

\section*{Paper 4}
\begin{description}
	\item[Title:]
	\item[Citation:] \cite{}
	\item[Abstract:]
	\item[Web link:] \url{}
	\item[Full text link:] \url{}
	\item[Comments:] 
\end{description}






\bibliographystyle{ieeetr}
\bibliography{comp150_agile}

\end{document}
