\documentclass{scrartcl}

\usepackage[hidelinks]{hyperref}
\usepackage[none]{hyphenat}

\title{Essay Proposal}
\subtitle{COMP150 - Agile Essay}

\author{Madeleine Kay}

\begin{document}

\maketitle

\section*{Topic}
The question I want to address in my essay is how can or should Agile be adapted for use in the game industry? Agile is designed for more commercial software with a team of programmers and a product owner. When a game is being made as well as programmers there are designers, artists, writers and many more people that need to be included to produce the game. Therefore Agile would require some adaptation to be used effectively in the games industry. 


\section*{Paper 1}
\begin{description}
	\item[Title:] 
	Agile Development Iterations and UI Design	
	\item[Citation:] \cite{Ferreira}
	\item[Abstract:] Many agile projects require user interaction (UI) design, but the integration of UI design into agile development is not well understood. This is because both agile development and UI design are iterative - but while agile methods iterate on code with iterations lasting weeks, UI designs typically iterate only on the user interface using low technology prototypes with iterations lasting hours or days. Similarly, both agile development and UI design emphasise testing, but agile development involves automated code testing, while UI must done by expert inspectors or ideally potential end users. We report on a qualitative grounded theory study of real agile projects involving significant UI design. The key results from our study are that agile iterations facilitates usability testing; allows software developers to incorporate results of those tests into subsequent iterations; and crucially, can significantly improve the quality of the relationship between UI designers and software developers.
	\item[Web link:] \url{http://ieeexplore.ieee.org/xpl/articleDetails.jsp?arnumber=4293575}
	\item[Full text link:] \url{http://agile2007.agilealliance.org/agile2007/downloads/proceedings/012_ferreria-AgileDevelopmentIterations-Final_585.pdf}
	\item[Comments:]  This paper looks at using Agile with designers as well as programmers and interacting the designers 
\end{description}


\section*{Paper 2}
\begin{description}
\item[Title:] Collaborative Events and Shared Artefacts: Agile Interaction Designers and Developers Working Toward Common Aims
\item[Citation:] \cite{Brown}
\item[Abstract:] Agile processes emphasize collaboration. We were interested in studying collaboration in agile teams including interaction designers, since the integration of user interaction design processes and software development processes is still an open issue. This study focused on designer and developer collaborations in the early stages of project work at four workplaces. We found designer-developer collaborations were extensive and we developed a categorization scheme of collaboration forms and artefacts that support this relationship. While some designer-developer collaborations were directed towards planning, which has been extensively researched, a larger part was directed towards realignment work. This latter type of collaborative work took three basic forms: scheduled, impromptu, and chats. Regardless of the form of collaboration, designer-developer interactions were mediated by twelve categories of artefacts. These artefacts helped designers and developers to determine, more specifically, what to create. We discuss the implications of our observations on alignment work for theory and practice.
\item[Web link:] \url{http://dl.acm.org/citation.cfm?id=2053950}
\item[Full text link:] Couldn't find one
\item[Comments:]  
\end{description}




\section*{Paper 3}
\begin{description}
	\item[Title:] Extreme programming and agile software development methodologies
	\item[Citation:] \cite{lindstrom2004extreme}
	\item[Abstract:] Several agile (i.e., lightweight) development methodologies, especially extreme programming 
	(XP), have been argued to be a solution to many of the problems that continue to plague soft-
	ware development projects. The authors provide a useful evaluation of such approaches, 
	including a discussion of the values that underlie the XP methodology
	\item[Web link:] \url{http://www.tandfonline.com/doi/abs/10.1201/1078/44432.21.3.20040601/82476.7?journalCode=uism20}
	\item[Full text link:] \url{http://www.nku.edu/~sakaguch/msis655/lindstrom2004.pdf}
	\item[Comments:]  This paper goes into a lot of detail on Agile and the XP method. It seems to contain a lot information that should be useful in my essay.
\end{description}

\section*{Paper 4}
\begin{description}
	\item[Title:] Developers' perspectives on iteration in game development
	\item[Citation:] \cite{Kultima}
	\item[Abstract:] In this article, the findings of an interview study on game developers' perspectives on iteration is presented. A two-part interview study conducted in 2010 and 2013--2014 suggests that from the perspective of game developers, iteration is an essential, natural and important part of the game development -- if not even a trivial part of the process. However, there is much to explore on the differences in practice and opinions revealing distinctive details in iterative processes potentially leading towards more elaborate design philosophies and methods on game development.
	\item[Web link:] \url{http://dl.acm.org.ezproxy.falmouth.ac.uk/citation.cfm?id=2818298&CFID=586277047&CFTOKEN=56353358}
	\item[Full text link:] \url{}
	\item[Comments:]  
\end{description}


\section*{Paper 5}
\begin{description}
	\item[Title:] 
	\item[Citation:] \cite{}
	\item[Abstract:] 
	\item[Web link:] \url{}
	\item[Full text link:]
	\item[Comments:]  
\end{description}

\bibliographystyle{ieeetr}
\bibliography{comp150_agile}

\end{document}
