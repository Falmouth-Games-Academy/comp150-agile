\documentclass{scrartcl}

\usepackage[hidelinks]{hyperref}
\usepackage[none]{hyphenat}

\title{Essay Proposal}
\subtitle{COMP150 - Agile Essay}

\author{Madeleine Kay}

\begin{document}

\maketitle

\section*{Topic}
The question I want to address in my essay is what is the most effective way to integrate the game design process into the Agile programming process. There seem to be many approaches to integrating design and programming from designing it all before hand to only designing what's needed in the sprint.

\section*{Paper 1}
\begin{description}
	\item[Title:] Agile Development Iterations and UI Design	
	\item[Citation:] \cite{Ferreira}
	\item[Abstract:] Many agile projects require user interaction (UI) design, but the integration of UI design into agile development is not well understood. This is because both agile development and UI design are iterative - but while agile methods iterate on code with iterations lasting weeks, UI designs typically iterate only on the user interface using low technology prototypes with iterations lasting hours or days. Similarly, both agile development and UI design emphasise testing, but agile development involves automated code testing, while UI must done by expert inspectors or ideally potential end users. We report on a qualitative grounded theory study of real agile projects involving significant UI design. The key results from our study are that agile iterations facilitates usability testing; allows software developers to incorporate results of those tests into subsequent iterations; and crucially, can significantly improve the quality of the relationship between UI designers and software developers.
	\item[Web link:] \url{http://ieeexplore.ieee.org/xpl/articleDetails.jsp?arnumber=4293575}
	\item[Full text link:] \url{http://agile2007.agilealliance.org/agile2007/downloads/proceedings/012_ferreria-AgileDevelopmentIterations-Final_585.pdf}
	\item[Comments:]  This paper looks at using iterative processes with designers and programmers. It looks at a series of real life projects and how they integrated the design and programming processes. It should be helpful for my essay as it looks at a few different ways of integrating UI design with Agile.
\end{description}


\section*{Paper 2}
\begin{description}
\item[Title:] Collaborative Events and Shared Artefacts: Agile Interaction Designers and Developers Working Toward Common Aims
\item[Citation:] \cite{Brown}
\item[Abstract:] Agile processes emphasize collaboration. We were interested in studying collaboration in agile teams including interaction designers, since the integration of user interaction design processes and software development processes is still an open issue. This study focused on designer and developer collaborations in the early stages of project work at four workplaces. We found designer-developer collaborations were extensive and we developed a categorization scheme of collaboration forms and artefacts that support this relationship. While some designer-developer collaborations were directed towards planning, which has been extensively researched, a larger part was directed towards realignment work. This latter type of collaborative work took three basic forms: scheduled, impromptu, and chats. Regardless of the form of collaboration, designer-developer interactions were mediated by twelve categories of artefacts. These artefacts helped designers and developers to determine, more specifically, what to create. We discuss the implications of our observations on alignment work for theory and practice.
\item[Web link:] \url{http://dl.acm.org/citation.cfm?id=2053950}
\item[Full text link:] Couldn't find one
\item[Comments:] The abstract says it's about designer and developer collaborations but I couldn't find a PDF so I'm not sure how relevant it will be. 
\end{description}



\section*{Paper 3}
\begin{description}
\item[Title:] Hitting the target: adding interaction design to agile software development
\item[Citation:] \cite{Patton}
\item[Abstract:] Extreme Programming appears to be a solution for discovering and meeting requirements faster (through close customer collaboration) as well as creating quality software. In practice we found XP did deliver high quality software quickly, but the resulting product still failed to delight the customer. Although the finished product should have been an exact fit, the actual end-user still ended up slogging through the system to accomplish necessary day-to-day work. This paper describes using interaction design in an agile development process to resolve this issue. Using interaction design as a day-to-day practice throughout an iterative development process helps our team at Tomax Technologies deliver high quality software, while feeling confident the resulting software will more likely meet end-user expectations. The method of Interaction Design followed here is based on Constantine and Lockwood's Usage-Centered Design. Recommendations are provided on how to practice an agile form of U-CD and how to incorporate bits of Interaction Design thinking into every day development and product planning decisions.
\item[Web link:] \url{http://dl.acm.org/citation.cfm?id=604255}
\item[Full text link:]
\item[Comments:]  I couldn't find a PDF, the abstract sounds like it's specific to the author's experience so may not be relevant but it does mention incorporating interaction design into Agile 
\end{description}



\section*{Paper 4}
\begin{description}
	\item[Title:] 	Incorporation of usability evaluation methods in agile software model	
	\item[Citation:] \cite{Butt}
	\item[Abstract:] Majority of agile projects currently involve interactive user interface designs, which can only be possible by following UCD in agile software model. But the integration of UCD is not clear in the current agile models. Both Agile models and UCD have iterative nature but agile models focus on coding and development of software; whereas, UCD focuses on user interface of the software. Similarly, both of them have testing features where the agile model involves automated tested code while UCD involves an expert or a user to test the user interface. In this paper, a new agile usability model is presented and tested in companies and are presented. Key results from these projects clearly shows: the proposed agile model incorporates usability evaluation methods, improve the relationship between usability experts to work with agile software experts; in addition, allows agile developers to incorporate the result from UCD into subsequent interactions.
	\item[Web link:] \url{http://ieeexplore.ieee.org/xpl/articleDetails.jsp?arnumber=7097336}
	\item[Full text link:] Couldn't find one
	\item[Comments:]  I couldn't find a PDF of this article but the abstract talks about user centred design and Agile and the authors propose a there own method combining the two.
\end{description}

\section*{Paper 5}
\begin{description}
	\item[Title:] The role of the interaction designer in an agile software development process
	\item[Citation:] \cite{Lievesley}
	\item[Abstract:] In this paper we describe observations of a contrast in thinking styles between a user-interface design team and a software engineering team developing a new software product. Presented in case study form, it is a first hand account by the interaction designers of work-in-progress. It concludes by identifying some key roles for the interaction designer working in an agile software development environment .
	\item[Web link:] \url{http://dl.acm.org/citation.cfm?id=1125647}
	\item[Full text link:] Couldn't find one
	\item[Comments:]  Again I couldn't find a PDF of this one, but the abstract mentions user interface design and software engineering but as it's a case study it may not be appropriate.
\end{description}


\section*{Paper 6}
\begin{description}
	\item[Title:] 
	\item[Citation:] \cite{}
	\item[Abstract:] 
	\item[Web link:] \url{}
	\item[Full text link:] \url{}
	\item[Comments:]  
\end{description}


\bibliographystyle{ieeetr}
\bibliography{comp150_agile}

\end{document}
