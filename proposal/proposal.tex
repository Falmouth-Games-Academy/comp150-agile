\documentclass[10pt,a4paper]{article}
\usepackage[latin1]{inputenc}
\usepackage{amsmath}
\usepackage{amsfonts}
\usepackage{amssymb}
\usepackage{cite}
\author{Connor Seán Rodgers}
\begin{document}
The question I am proposing to answer is, "How does version control facilitate the scrum process?".  I will be approaching this by looking at the scrum process, in particular sprints and how they can and cannot work with version control processes. The scrum framework works around a task board and daily scrum meetings which outline the tasks that are being done, need to be done and are done on the current "sprint" \cite{rubart2009supporting}. Sprints are a designated amount of time to complete a userstory, or aspect of one, for the game or product and are set a deadline based on an approximate time frame \cite{dinakar2009agile} I wish to explore how the use of branches \cite {phillips2011branching} and some other version control practices can be used to tie version control directly to the scrum process. The use of version control can allow different methods to be explored for a single user story without interfering with the main branch of the software\cite {mikami2017micro}. I will be focusing my research on branching as in my opinion this will greatly lend itself to the scrum process\cite{rayana2016gitwaterflow}\cite{krusche2016teaching}.

\bibliography{references}{}
\bibliographystyle{plain}
\end{document}2