\documentclass{scrartcl}

\usepackage[hidelinks]{hyperref}
\usepackage[none]{hyphenat}

\title{How can different personality types best adapt to the agile philosophy?}
\subtitle{COMP150 - Agile Philosophy Essay}

\author{Hannah Utley}

\begin{document}

\maketitle

\section*{Topic}

D.Bishop and A.Deokar have reported a correlation between an individual’s personality traits and their preference for agile processes 
\cite{Paper1}. In their samples, those who were more open and extraverted had a greater preference for agile. Whereas neurotic 
personality types (or emotional instability) were less likely to prefer agile methods. I would like to explore how individuals, with 
personality types that have been found to have a negative relationship with agile preference, can best adapt to using such processes. A 
variety of personalities and skill-sets in individuals within a team tends to enhance performance \cite{Paper3}, therefore it is 
important not to disregard those who struggle with agile principles. Finding methods that give individuals the autonomy to align their 
working style with agile, may help ensure diversity within teams.

\section*{Paper 1}
\begin{description}
\item[Title:] Toward an Understanding of Preference for Agile Software Development Methods from a Personality Theory Perspectiv'
\item[Citation:] \cite{Paper1}
\item[Abstract:] This paper presents the results of an exploratory research study that investigates factors contributing to preference 
for the agile software development approaches. The initial exploration revolves around the Five Factor Model of personality and the 
premise that these personality factors provide a partial explanation of preference for an agile approach. A survey instrument for 
measuring the preference for agile methods was developed and validated. The results from the quantitative data collected from the survey 
study indicate that three out of the five personality factors from the Five Factor Model show a correlation with above average 
preference for agile methods. These factors are extra version, openness and neuroticism. The first two have a positive relationship with 
agile preference while neuroticism (emotional instability) has a negative relationship with agile methodology preference. To further 
investigate the results, an exploratory factor analysis was performed on the data, which identified three factors that may also 
contribute to a preference for agile methods.
\item[Web link:] \url{http://ieeexplore.ieee.org/document/6759185/}
\item[Full text link:] \url{http://ieeexplore.ieee.org/stamp/stamp.jsp?arnumber=6759185}
\item[Comments:] 
\end{description}

\section*{Paper 2}
\begin{description}
\item[Title:] Agile Practices: The Impact on Trust in Software Project Teams
\item[Citation:] \cite{Paper2}
\item[Abstract:] {Agile software development involves self-managing teams that are empowered and responsible for meeting project goals 
in whatever way they deem suitable. Managers must place more trust in such teams than they do in teams following more traditional 
development methodologies. The authors highlight how the use of agile practices can enhance trust amongst agile team members. They also 
present challenges that agile teams can face as a result of using agile practices. Their results are based on the findings from three 
case studies of agile software development teams.}
\item[Web link:] \url{http://ieeexplore.ieee.org/document/6007124/}
\item[Full text link:] \url{http://ieeexplore.ieee.org.ezproxy.falmouth.ac.uk/stamp/stamp.jsp?arnumber=6007124}
\item[Comments:] Write a few sentences on how you found the article and why you believe it is relevant and/or important.
\end{description}

\section*{Paper 3}
\begin{description}
\item[Title:] Supporting agile team composition: A prototype tool for identifying personality (In)compatibilities
\item[Citation:] \cite{Paper3}
\item[Abstract:] {Extensive work in the behavioral sciences tells us that team composition is a complex activity in many disciplines, 
given the variations inherent across individuals' personalities. The composition of teams to undertake software development is subject 
to this same complexity. Furthermore, the building of a team to undertake agile software development may be particularly challenging, 
given the inclusive yet fluid nature of teams in this context. We describe here the development and preliminary evaluation of a 
prototype tool intended to assist software engineers and project managers in forming agile teams, utilizing information concerning 
members' personalities as input to this process. Initial assessment of the tool's capabilities by agile development practitioners 
suggests that it would be of value in supporting the team composition activity in real projects.}
\item[Web link:] \url{http://ieeexplore.ieee.org.ezproxy.falmouth.ac.uk/document/5071413/}
\item[Full text link:] \url{http://ieeexplore.ieee.org.ezproxy.falmouth.ac.uk/stamp/stamp.jsp?arnumber=5071413}
\item[Comments:] Write a few sentences on how you found the article and why you believe it is relevant and/or important.
\end{description}

\section*{Paper 4}
\begin{description}
\item[Title:] {Critical personality traits in successful pair programming}
\item[Citation:] \cite{Paper4}
\item[Abstract:]{ Pair programming (PP) is a common practice in Extreme programming, in which two programmers work together using a 
single computer. The close interaction required by PP makes it difficult to apply. The hypothesis is that certain personality traits are 
crucial for the success of PP, and PP partners should be chosen based on these personality traits. In this research, we first survey the 
programmers in industry to identify the perceived important personality traits for PP, and then conduct experiments to determine the 
significance of these personality traits in successful PP.}
\item[Web link:] \url{http://ieeexplore.ieee.org.ezproxy.falmouth.ac.uk/document/1667566/}
\item[Full text link:] \url{http://ieeexplore.ieee.org.ezproxy.falmouth.ac.uk/stamp/stamp.jsp?arnumber=1667566}
\item[Comments:] Write a few sentences on how you found the article and why you believe it is relevant and/or important.
\end{description}

\section*{Paper 5}
\begin{description}
\item[Title:] {Personality Matters: Balancing for Personality Types Leads to Better Outcomes for Crowd Teams}
\item[Citation:] \cite{Paper5}
\item[Abstract:]{ When personalities clash, teams operate less effectively. Personality differences affect face-to-face collaboration 
and may lower trust in virtual teams. For relatively short-lived assignments, like those of online crowdsourcing, personality matching 
could provide a simple, scalable strategy for effective team formation. However, it is not clear how (or if) personality differences 
affect teamwork in this novel context where the workforce is more transient and diverse. This study examines how personality 
compatibility in crowd teams affects performance and individual perceptions. Using the DISC personality test, we composed 14 five-person 
teams (N=70) with either a harmonious coverage of personalities (balanced) or a surplus of leader-type personalities (imbalanced). 
Results show that balancing for personality leads to significantly better performance on a collaborative task. Balanced teams exhibited 
less conflict and their members reported higher levels of satisfaction and acceptance. This work demonstrates a simple personality 
matching strategy for forming more effective teams in crowdsourcing contexts.}
\item[Web link:] \url{http://dl.acm.org.ezproxy.falmouth.ac.uk/citation.cfm?id=2819979&CFID=859129870&CFTOKEN=96109321#}
\item[Full text link:] \url{http://delivery.acm.org.ezproxy.falmouth.ac.uk/10.1145/2820000/2819979/260_lykourentzou.pdf?ip=193.61.64.8&id=2819979&acc=ACTIVE%20SERVICE&key=BF07A2EE685417C5%2EEAA225A8AB01C582%2E4D4702B0C3E38B35%2E4D4702B0C3E38B35&CFID=859129870&CFTOKEN=96109321&__acm__=1477844115_7a84d94d877b58fc490e1c7c0f15cbcb}
\item[Comments:] Write a few sentences on how you found the article and why you believe it is relevant and/or important.
\end{description}

\bibliographystyle{ieeetran}
\bibliography{agileessaybibliography}

\end{document}
