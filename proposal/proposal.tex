\documentclass{scrartcl}

\usepackage[hidelinks]{hyperref}
\usepackage[none]{hyphenat}

\title{Essay Proposal}
\subtitle{COMP150 - Agile Development Essay}

\author{Alexander Mitchell}

\begin{document}

\maketitle

\section*{Topic}

My essay will be on: The impact of the Agile philosopy to the morale of a team across various stages in a development cycle.

% Add details as appropriate.

\section*{Paper 1}
% This is an example! Replace the details with a paper relevant to your chosen topic.
\begin{description}
\item[Title:] Enhancing the Performance of Software Development Virtual Teams through the Use of Agile Methods: A Pilot Study
\item[Citation:] \cite{bibtex_key}
\item[Abstract:] ``This paper develops a conceptual model that
explicates the role of synchronous communication
media in enabling – directly and indirectly, via social
presence – virtual software development teams to
adopt and apply Agile methods. In turn, Agile
methods, as well as perceived social presence, are
theorized to have a positive impact on
communication convergence and transactive
memory. Ultimately, these outcomes are formulated
as direct antecedents of virtual team performance. A
pilot study of 40 Free/Libre Open Source Software
(FLOSS) teams provides preliminary supporting
evidence for the conceptual model.''
\item[Web link:] \url{https://www.computer.org/csdl/proceedings/hicss/2011/4282/00/01-09-03.pdf}
\item[Full text link:] \url{https://www.computer.org/csdl/proceedings/hicss/2011/4282/00/01-09-03.pdf}
\item[Comments:]
\end{description}

\section*{Paper 2}
\begin{description}
\item[Title:] Agile Lessons from Ryse and Crysis 3
\item[Citation:] \cite{bibtex_key}
\item[Abstract:] 
\item[Web link:] http://www.gdcvault.com.ezproxy.falmouth.ac.uk/play/1020790/Agile-Lessons-from-Ryse-and
\item[Full text link:] 
\item[Comments:] 
\end{description}

\section*{Paper 3}
\begin{description}
\item[Title:] Agile Methods for Large Organizations - Building Communities of Practice
\item[Citation:] \cite{bibtex_key}
\item[Abstract:] Agile development practices respect tacit knowledge, make communication more effective, and thus foster the knowledge creation process. However the current agile methods, like XP, are focused on practices that individual teams or projects need, and the use of the methods in organizations consisting of multiple cooperating teams is difficult. The Community of Practice theory suggests that large agile organizations should have various overlapping, informal cross-team communities. This paper studies three agile methods developed at Nokia that use facilitated workshops to solve multi-team issues. The paper explains using Communities of Practices theory - why these methods work in multi-team settings. The results of this paper suggest that workshop practices that amass people from different parts of organizations to perform a specific well-defined task can be used effectively to solve issues that span over multiple teams and to build up Communities of Practice. This result suggests that the Community of Practice concept could provide a basis for adapting agile methods for the needs of large organizations.
\item[Web link:] https://www.computer.org/csdl/proceedings/adc/2004/2248/00/22480002.pdf
\item[Full text link:] 
\item[Comments:] 
\end{description}

\section*{Paper 4}
\begin{description}
\item[Title:] Improving Software Process in Agile Software Development Projects: Results from Two XP Case Studies
\item[Citation:] \cite{bibtex_key}
\item[Abstract:] One of the Agile principles is that software development teams should regularly reflect on how to improve their practices to become more effective. Some systematic approaches have been proposed on how to conduct such a self-reflection process, but little empirical evidence yet exits. In this paper, the empirical results are reported from two XP (Extreme Programming) projects where the project teams conducted "post-iteration workshops" after all process iterations in order to improve and optimize working methods. Both qualitative and quantitative data from the total of eight post-iteration workshops is presented in order to evaluate and compare the findings of the two projects. The results show the decline of both positive and negative findings, as well as the narrower variation of negative findings and process improvement actions towards the end of both projects. In both projects, the data from post-iteration workshops indicate increased satisfaction and learning of project teams.
\item[Web link:] https://www.computer.org/csdl/proceedings/euromicro/2004/2199/00/21990310.pdf
\item[Full text link:] 
\item[Comments:] 
\end{description}

\section*{Paper 5}
\begin{description}
\item[Title:] Scrum + Engineering Practices: Experiences of Three Microsoft Teams
\item[Citation:] \cite{bibtex_key}
\item[Abstract:] The Scrum methodology is an agile software development process that works as a project management wrapper around existing engineering practices to iteratively and incrementally develop software. With Scrum, for a developer to receive credit for his or her work, he or she must demonstrate the new functionality provided by a feature at the end of each short iteration during an iteration review session. Such a short-term focus without the checks and balances of sound engineering practices may lead a team to neglect quality. In this paper we present the experiences of three teams at Microsoft using Scrum with an additional nine sound engineering practices. Our results indicate that these teams were able to improve quality, productivity, and estimation accuracy through the combination of Scrum and nine engineering practices.
\item[Web link:] http://ieeexplore.ieee.org.ezproxy.falmouth.ac.uk/document/6092605/
\item[Full text link:] http://ieeexplore.ieee.org.ezproxy.falmouth.ac.uk/document/6092605/
\item[Comments:] 
\end{description}

\section*{Paper 6}
\begin{description}
\item[Title:] Team Performance in Software Development: Research Results versus Agile Principles
\item[Citation:] \cite{bibtex_key}
\item[Abstract:] This article reviews scientific studies of factors influencing colocated development teams' performance and proposes five factors that strongly affect performance. In the process, it compares these propositions with the Agile Manifesto's development principles. The Web extra at https://extras.computer.org/extra/mso2016040106s1.pdf details the sources and research methods the authors employed.
\item[Web link:] http://ieeexplore.ieee.org.ezproxy.falmouth.ac.uk/document/7498535/
\item[Full text link:] http://ieeexplore.ieee.org.ezproxy.falmouth.ac.uk/xpls/icp.jsp?arnumber=7498535
\item[Comments:] 
\end{description}

\bibliographystyle{ieeetran}
\bibliography{initial_references}

\end{document}
