\documentclass{scrartcl}

\usepackage[hidelinks]{hyperref}
\usepackage[none]{hyphenat}

\title{Essay Proposal}
\subtitle{COMP110 - Computer Architecture Essay}

\author{Michail Karakasis}

\begin{document}

\maketitle

\section*{Topic}

My essay will be on:

I will be answering the following research question: How does running daily stand-ups in a virtual office setting, using tools such as Slack, influence time-boxing and team productivity? I will be focusing on how virtual tools influence time-boxing and team productivity, the importance of using virtual tools and the various effects they evoke on different people. I have chosen to explore this question because I feel like I have enough personal experience to discuss about this certain topic, something that I think will reinforce my facts and opinions. Additionally, I believe it to be an especially important issue in today's world of business and technology, where we can merge both to provide and receive good and realistically timed products.

\section*{Paper 1}
\begin{description}
\item[Title:] Virtual Offices
\item[Citation:] \cite{shannon}
\item[Abstract:] ``A concept for a new business communication environment consisting of a shared and interactive multiuser virtual space that consists of a computer-graphics-based virtual space and video-based objects is proposed. This combination results in an extremely flexible environment for remote human collaboration. The human interface, which entails walking and interacting within the virtual space and offers a new dimension in daily communication activities, is described.''
\item[Web link:] \url{http://ieeexplore.ieee.org.ezproxy.falmouth.ac.uk/document/237980/}
\item[Full text link:] \url{http://ieeexplore.ieee.org.ezproxy.falmouth.ac.uk/stamp/stamp.jsp?tp=&arnumber=237980}
\item[Comments:] Write a few sentences on how you found the article and why you believe it is relevant and/or important.
\end{description}

\section*{Paper 2}
\begin{description}
\item[Title:] On measuring programmer team productivity
\item[Citation:] \cite{bibtex_key}
\item[Abstract:] ``It is difficult to study industrial programmer productivity because of the extreme variance seen among individual programmers and the difficulty of performing controlled experiments. As an alternative to studying individual programmers, we examine the group productivity of programmer teams. We postulate that there is such a thing as average programmer productivity, in a given context. By studying programmer teams, we can eventually obtain measures of the expected performance of an average programmer in a defined context. Differences in project productivity can then be attributed to process characteristics. Existing project data is examined to see how data could be collected to support the idea of a standard programmer.''
\item[Web link:] \url{http://ieeexplore.ieee.org.ezproxy.falmouth.ac.uk/document/685594/}
\item[Full text link:] \url{http://ieeexplore.ieee.org.ezproxy.falmouth.ac.uk/stamp/stamp.jsp?tp=&arnumber=685594}
\item[Comments:] Write a few sentences on how you found the article and why you believe it is relevant and/or important.
\end{description}

\section*{Paper 3}
\begin{description}
\item[Title:] Workplace collaboration in a 3D Virtual Office
\item[Citation:] \cite{bibtex_key}
\item[Abstract:] ``We describe Virtual Office, an innovative application of virtual world technology for enabling informal office interactions and collaboration even when some of the participants are physically out of office. Each instance of the system is tied to an actual physical office, so the communication and visual channels created among its users are designed to offer the level of privacy in the corresponding real-world office. VirtualOffice supports auras and automated navigation based on logical seats in the office, rather than geometric distances. The system is implemented using a distributed MVC architecture employing a practical combination of (a) push and pull communication, and (b) cloud-based servers. The system is designed to support remote `management by walking around as well as virtually visiting both collaborators' and ones' own offices, thereby enabling informal conversations that seamlessly bridge the physical and virtual worlds. VirtualOffice also represents a new point in both Benford's and Schroeder's taxonomies of collaboration systems that classifies instant messaging, virtual worlds, and video conferencing. A detailed scenario is used to motivate our new design point and compare it with commonly used as well as emerging collaboration applications as well as established virtual worlds such as Second Life, for the specific purpose of informal office collaboration.''
\item[Web link:] \url{http://ieeexplore.ieee.org.ezproxy.falmouth.ac.uk/document/5759582/}
\item[Full text link:] \url{http://ieeexplore.ieee.org.ezproxy.falmouth.ac.uk/stamp/stamp.jsp?arnumber=5759582}
\item[Comments:] Write a few sentences on how you found the article and why you believe it is relevant and/or important.
\end{description}

\section*{Paper 4}
\begin{description}
\item[Title:] Back to Basics: The Role of Agile Principles in Success with an Distributed Scrum Team
\item[Citation:] \cite{bibtex_key}
\item[Abstract:] ``Agile processes rely on feedback and communication to work and they often work best with co-located teams for this reason. Sometimes agile makes sense because of project requirements and a distributed team makes sense because of resource constraints. A distributed team can be productive and fun to work on if the team takes an active role in overcoming the barriers that distribution causes. This is the story of how one product team used creativity, communications tools, and basic good engineering practices to build a successful product.''
\item[Web link:] \url{http://ieeexplore.ieee.org.ezproxy.falmouth.ac.uk/document/4293626/}
\item[Full text link:] \url{http://ieeexplore.ieee.org.ezproxy.falmouth.ac.uk/stamp/stamp.jsp?arnumber=4293626}
\item[Comments:] Write a few sentences on how you found the article and why you believe it is relevant and/or important.
\end{description}

\section*{Paper 5}
\begin{description}
\item[Title:] Virtual meetings: not just an option any more!
\item[Citation:] \cite{bibtex_key}
\item[Abstract:] ``How much time and money do people spend going to, attending, and returning from meetings? How much of those resources are well spent? Global enterprises have a long history of collaboration across geographic distance. They are pushing to optimize the value of their meetings. While virtual meetings are not entirely new, a new model with new processes is required to optimize their value. This abstract represents my perspective on virtual meetings after participating in a study conducted by the Department of Anthropology at the University of North Texas. It compares and contrasts face-to-face meetings with virtual ones in the areas of performance, virtual and physical space, the plurality of spaces and their subsequent interactions, culture, relationships, benefits, and barriers. These and related questions are discussed. Remember the orange juice commercial, "It's not just for breakfast any more!" So too, is the virtual meeting: "It's not just an option any more, it's an active preferred choice".''
\item[Web link:] \url{http://ieeexplore.ieee.org.ezproxy.falmouth.ac.uk/document/1252311/}
\item[Full text link:] \url{http://ieeexplore.ieee.org.ezproxy.falmouth.ac.uk/stamp/stamp.jsp?arnumber=1252311}
\item[Comments:] Write a few sentences on how you found the article and why you believe it is relevant and/or important.
\end{description}

\section*{Paper 6}
\begin{description}
\item[Title:] Improving Project Communication with Virtual Team Boards
\item[Citation:] \cite{bibtex_key}
\item[Abstract:] ``User Stories are an important artifact in agile projects. Good understanding of user stories is crucial for project success. Distributed software teams rely on tools as a means for their communication about user stories. Current tools, however, fix the structure and visualization of stories and thereby limit potential information flow. We propose a concept that enables the teams to decide by themselves how they want to visualize their set of user stories. We want to investigate the use of the concept, its efficiency and the applicability to agile distributed teams.''
\item[Web link:] \url{http://ieeexplore.ieee.org.ezproxy.falmouth.ac.uk/document/6337316/}
\item[Full text link:] \url{http://ieeexplore.ieee.org.ezproxy.falmouth.ac.uk/stamp/stamp.jsp?arnumber=6337316}
\item[Comments:] Write a few sentences on how you found the article and why you believe it is relevant and/or important.
\end{description}

\section*{Paper 7}
\begin{description}
\item[Title:] Agile Practices: The Impact on Trust in Software Project Teams
\item[Citation:] \cite{bibtex_key}
\item[Abstract:] ``Agile software development involves self-managing teams that are empowered and responsible for meeting project goals in whatever way they deem suitable. Managers must place more trust in such teams than they do in teams following more traditional development methodologies. The authors highlight how the use of agile practices can enhance trust amongst agile team members. They also present challenges that agile teams can face as a result of using agile practices. Their results are based on the findings from three case studies of agile software development teams.''
\item[Web link:] \url{http://ieeexplore.ieee.org.ezproxy.falmouth.ac.uk/document/6007124/}
\item[Full text link:] \url{http://ieeexplore.ieee.org.ezproxy.falmouth.ac.uk/stamp/stamp.jsp?arnumber=6007124}
\item[Comments:] Write a few sentences on how you found the article and why you believe it is relevant and/or important.
\end{description}


\bibliographystyle{ieeetran}
\bibliography{initial_references}

\end{document}
