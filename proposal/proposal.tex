\documentclass{scrartcl}

\usepackage[hidelinks]{hyperref}
\usepackage[none]{hyphenat}

\title{Essay Proposal}
\subtitle{COMP150 - Agile Devolopment Essay}

\author{Jack Maber}

\begin{document}

\maketitle

\section*{Topic}

My essay will be on:

% Add details as appropriate.

\section*{Paper 1}
% This is an example! Replace the details with a paper relevant to your chosen topic.
\begin{description}
\item[Title:] Is agility out there?: agile practices in game development
\item[Citation:] \cite{shannon}
\item[Abstract:] ``Game development is a very complex and multidisciplinary activity and surely the success of games as one of most profitable areas in entertainment domain could not be incidentally. The goal of this paper is to investigate if (and how) principles and practices from Agile Methods have been adopted in game development, mainly gathering evidences through Postmortem Analysis (PMA).

Then we describe how we have conducted PMA in order to identify the good practices adopted in several game development projects. The results are discussed, comparing similarities and differences on how these practices are taken in account in (traditional) software development and game development.''
\item[Web link:] \url{http://dl.acm.org.ezproxy.falmouth.ac.uk/citation.cfm?id=1878453}
\item[Full text link:] \url{http://dl.acm.org.ezproxy.falmouth.ac.uk/ft_gateway.cfm?id=1878453&type=pdf&CFID=859269260&CFTOKEN=64719971}
\item[Comments:] This paper investigates how agile principles and practice are applied in game development. Also states that having a qualified team is the where most good practice will take place.
\end{description}

\section*{Paper 2}
\begin{description}
\item[Title:] Towards Agent-based Agile approach for Game Development Methodology
\item[Citation:] \cite{bibtex_key}
\item[Abstract:] Game development is very complex and the success of the game is based on the game development methods. The purpose of this paper is to investigate on the existing game development methods and provide an upcoming game development method that is based on predictive and adaptive development models. A critical analysis to Agile method which are mostly used in modern game development methods is presented. We identified the weakness of Agile game development and solve it by creating a cooperation with Agent Oriented Software Engineering (AOSE) to introduce a new hybrid methodology named as Agent Agile Game Development Methodology (AAGDM) that combines both predictive and adaptive models.
\item[Web link:] \url{http://ieeexplore.ieee.org.ezproxy.falmouth.ac.uk/document/6916626/}
\item[Full text link:] \url{http://ieeexplore.ieee.org.ezproxy.falmouth.ac.uk/stamp/stamp.jsp?arnumber=6916626}
\item[Comments:] This paper presents a critical analysis of the Agile Devolopment process currently used in today's games industry, identifying the weaknesses that are present.
\end{description}

\section*{Paper 3}
\begin{description}
\item[Title:] Prioritising user stories in agile environment
\item[Citation:] \cite{bibtex_key}
\item[Abstract:] In the last few years Agile methodologies appeared as a reaction to traditional software development methodologies. In Agile environment the requirements from the client are always taken in the form of user-stories and prioritization of requirements is done by Moscow method, validate learning and walking skeleton methods. By literature survey it has been observed that these methods are not efficient because they do not consider importance of user-stories by client. In this research work some importance related and effort related factors are considered on the basis of which the prioritization of user-stories is done. Further the feasibility of work has been validated by a case study of Enable Quiz which is a lightweight technical quizzing solution; for companies that hire engineers. The research work will allows the companies to better screen job candidates and assess their internal talent for skills development.
\item[Web link:]\url {http://ieeexplore.ieee.org.ezproxy.falmouth.ac.uk/document/6781336/}
\item[Full text link:] \url{http://ieeexplore.ieee.org.ezproxy.falmouth.ac.uk/stamp/stamp.jsp?arnumber=6781336}
\item[Comments:]  
\end{description}

\section*{Paper 4}
\begin{description}
\item[Title:] How to reduce user story reopen count in Scrum development?
\item[Citation:] \cite{bibtex_key}
\item[Abstract:] A user story (US) is reopened for reworking due to shortcomings from four major fronts- business analyst (BA), developer, quality analyst (QA) and environmental issues. BA is responsible for capturing requirements and documenting the requirements in the form of user stories; developer is responsible for the implementation of the user story; and the QA is responsible for testing of US. Now if any of three does not perform his job accurately then the probability of reopening of US increases. So we can reduce the probability of reopening of US by improving on shortcoming from three ends BA, developer and QA. As for as environmental issues, are concerned, they can be controlled by QA, developer and BA. The aim of the paper is to identify different areas from BA, Developer and QA's end to reduce the probability of reopening of a US and thereby reducing the user story reopen count in the Scrum development.
\item[Web link:] http://ieeexplore.ieee.org.ezproxy.falmouth.ac.uk/document/7100629/
\item[Full text link:] http://ieeexplore.ieee.org.ezproxy.falmouth.ac.uk/stamp/stamp.jsp?arnumber=7100629
\item[Comments:] Write a few sentences on how you found the article and why you believe it is relevant and/or important.
\end{description}

\section*{Paper 5}
\begin{description}
\item[Title:] Title of paper
\item[Citation:] \cite{bibtex_key}
\item[Abstract:] Copy and paste the abstract here
\item[Web link:] Give the URL of the paper in IEEE Xplore, ACM Digital Library, or similar
\item[Full text link:] Give the URL of a downloadable PDF of the paper, if you can find one
\item[Comments:] Write a few sentences on how you found the article and why you believe it is relevant and/or important.
\end{description}

\section*{Paper 6}
\begin{description}
\item[Title:] Title of paper
\item[Citation:] \cite{bibtex_key}
\item[Abstract:] Copy and paste the abstract here
\item[Web link:] Give the URL of the paper in IEEE Xplore, ACM Digital Library, or similar
\item[Full text link:] Give the URL of a downloadable PDF of the paper, if you can find one
\item[Comments:] Write a few sentences on how you found the article and why you believe it is relevant and/or important.
\end{description}

\bibliographystyle{ieeetran}
\bibliography{initial_references}

\end{document}
