\documentclass{scrartcl}

\usepackage[hidelinks]{hyperref}
\usepackage[none]{hyphenat}

\title{Essay Proposal}
\subtitle{COMP150 - Agile Essay}

\author{JH182233}

\begin{document}


\maketitle

\section*{Topic}

My essay will be on

Would The Agile Methodology Benefit From Having Greater Command And Control Influences?

\section*{Paper 1}
\begin{description}
\item[Title:] Ad-hoc leadership in agile software development environments
\item[Citation:] \cite{Leader}
\item[Abstract:] ``Leadership is the ability to influence people, leading them to behave in a certain way in order to achieve the group's goals. Leadership is independent of job titles and descriptions. Usually, however, in order to lead, leaders need the power derived from their organizational positions. There are different leadership styles, like task-oriented versus people-oriented, directive versus permissive, autocrat versus democrat. In this paper, we examine the leadership concept in software development environments and focus on leadership in transition processes to agile software development. Specifically, based on our comprehensive research on agile software development, we suggest a leadership style - ad-hoc leadership - that usually emerges in such change processes. We present the characteristics, dynamic and uniqueness of this leadership style and illustrate its usefulness for the analysis of representative scenarios.''
\item[Web link:] \url{http://dl.acm.org/citation.cfm?id=1833316&CFID=757152411&CFTOKEN=34331036}
\item[Full text link:] \url{None}
\item[Comments:] This paper seems to look further into leadership and the concepts behind it, it would be useful for defining what it it and how to use it with agile in the essay.
\end{description}

\section*{Paper 2}
\begin{description}
	\item[Title:] Agile principles as a leadership value system in the software development: are we ready to be unleashed?
	\item[Citation:] \cite{LeaderUnleashed}
	\item[Abstract:] ``Agile methods generally promote a disciplined project management process that encourages frequent inspection and adaptation, a leadership philosophy that encourages teamwork, self-organization and accountability, a set of engineering best practices that allow for rapid delivery of high-quality software, and a business approach that aligns development with customer needs and company goals. When it is considered Agile Principles as a leadership value System in the Software Development then the question arise that are one is ready to be unleashed? As criticisms include several issues regarding the same like Agile Principles often used as a means to bleed money from customers through lack of defining a deliverable, Lack of structure and necessary documentation, Only works with senior-level developers Incorporates insufficient software design, Requires meetings at frequent intervals at enormous expense to customers, Requires too much cultural change to adopt, Can lead to more difficult contractual negotiations?, Can be very inefficient---if the requirements for one area of code change through various iterations, the same programming may need to be done several times over. Impossible to develop realistic estimates of work effort needed to provide a quote, because at the beginning of the project no one knows the entire scope/requirements and can increase the risk of scope due to the lack of detailed requirements documentation?
	
	Here it is observed and surveyed the various categories of Projects for different kind of group members to adopt and follow the agile principles. It proposes the strength and weaknesses of all 12 Agile Principles based on Indian scenario.
	''
	\item[Web link:] \url{http://dl.acm.org/citation.cfm?id=1980188&CFID=757152411&CFTOKEN=34331036}
	\item[Full text link:] \url{None}
	\item[Comments:] This paper seems to look into management styles along with scope of the project this might be helpful when it comes to explaining about team members slacking on their work.
\end{description}

\section*{Paper 3}
\begin{description}
	\item[Title:] An empirical study into social success factors for agile software development
	\item[Citation:] \cite{SocialAgile}
	\item[Abstract:] ``Though many warn that Agile at larger scale is problematic or at least more challenging than in smaller projects, Agile software development seems to become the norm, also for large and complex projects.
	
	Based on literature and qualitative interviews, we constructed a conceptual model of social factors that may be of influence on the success of software development projects in general, and of Agile projects in particular. We also included project size as a candidate success factor.
	
	We tested the model on a set of 40 projects from 19 Dutch organizations, comprising a total of 141 project members, Scrum Masters and product owners.
	
	We found that project size does not determine Agile project success. Rather, value congruence, degree of adoption of Agile practices, and transformational leadership proved to be the most important predictors for Agile project success.
	''
	\item[Web link:] \url{http://dl.acm.org/citation.cfm?id=2819335&CFID=757152411&CFTOKEN=34331036}
	\item[Full text link:] \url{https://www.sig.eu/files/9714/2375/2662/KellePlaatWijstVisser-2014.pdf}
	\item[Comments:] Here it appears to go into the scale of the project and its success based on the leadership of the team which would be useful for defining the scrum masters responsibilities and affect on bigger projects.
\end{description}

\section*{Paper 4}
\begin{description}
	\item[Title:] Exploring agile
	\item[Citation:] \cite{Agile}
	\item[Abstract:] ``Bob attempts to figure out exactly what Agile development is and where it works by analyzing the Agile Manifesto and its supporting principles. He proposes changes to some troubling statements.''
	\item[Web link:] \url{http://dl.acm.org/citation.cfm?id=1370144&CFID=757152411&CFTOKEN=34331036}
	\item[Full text link:] \url{http://clearspecs.com/joomla15/downloads/ClearSpecs59V01_Exploring%20Agile.pdf}
	\item[Comments:] This will be usefull for explaining Agile.
\end{description}

\section*{Paper 5}
\begin{description}
	\item[Title:] Agile project manager behavior: The taxonomy 
	\item[Citation:] \cite{Behavior}
	\item[Abstract:] ``The past few years have witnessed dynamic changes in the field of software project management. These provided evidence of the strength of agile methodologies as a strategy that can speed up the development of rhythms and growth of innovation. Managing a project involves a complexity of requirements and developmental processes. This provides challenges to the project manager as he is accountable for the failure of a completed project. As such he is required to tackle any problems by adopting agile methodology during the development process. In Malaysia there is little research done to examine the behavior of the agile project manager. This study aims to analyze the behavior when project managers adopt agile in managing the development projects. Moreover, a review of the relevant literature has helped in developing an understanding of the agile project manager's behavior as this is necessary for them to become more agile inside development projects. For that reason, this paper has identified seven behaviors that the manager needs to adopt during software development process. These include Leadership, Openness, Results Orientation, Ethics, Communication, Strategic and Creative and Innovative. This paper contributes to the relevant theory by developing taxonomy of the agile project manager's behavior. Practitioners can use this taxonomy as a sensitizing device that helps the manager to consider behavior that promotes success of their projects. This is important to ensure an increase productivity and profitability, which are the business strategies of software development projects.''
	\item[Web link:] \url{http://ieeexplore.ieee.org/xpl/articleDetails.jsp?arnumber=6986020&queryText=agile%20manager&newsearch=true}
	\item[Full text link:] \url{None}
	\item[Comments:] I believe this explains the transition of manager to scrum master regarding thier behaviour and its affects on the project, wich would be useful to explain why command and control is needed.
\end{description}

\section*{Paper 6}
\begin{description}
	\item[Title:] Adoption of agile methodology in software development 
	\item[Citation:] \cite{AdoptAgile}
	\item[Abstract:] ``As adopting Agile software development becomes a trend, there is a need for a more structured definition of what is Agile and what is a high-level of Agile maturity. Traditional development methodologies rely on documents to record and pass on knowledge from one specialist to the next. Feedback cycles are, in many cases, too long or even nonexistent. Agile principles emphasize building working software that people can get hands on quickly, versus spending a lot of time writing specifications up front. Agile development focuses on cross-functional teams empowered to make decisions, versus big hierarchies and splitting by function. It also focuses on rapid iteration, with continuous customer input along the way. This paper deals with Agile methodology and scaling. The special highlight is put on people investigating their contribution in Agile approach success. Some reflections after using Agile in our own organization are also presented.''
	\item[Web link:] \url{http://ieeexplore.ieee.org/xpl/articleDetails.jsp?arnumber=6596295&newsearch=true&queryText=what%20is%20agile}
	\item[Full text link:] \url{https://bib.irb.hr/datoteka/630342.Agile.pdf}
	\item[Comments:] This paper seems to look into the difference between agile and a traditional method, this can be usful to compaer the productivity of the two methods.
\end{description}

\section*{Paper 7}
\begin{description}
	\item[Title:] A Product Manager's Guide to Surviving the Big Bang Approach to Agile Transitions 
	\item[Citation:] \cite{ManagerGuide}
	\item[Abstract:] ``In 2006, Salesforce.com took a big bang approach to implementing the Adaptive Development Methodology (ADM): a unique blend of Scrum-style project management and the agile software development framework. The result was a huge success. In just three months, an RandD department consisting of 200 employees and 30 teams transitioned from a traditional waterfall-style approach to ADM. Less than sixteen months later, the company had realized a 37 percent increase in productivity.The rapid transition of ADM precipitated massive changes in how product managers work, interact with members of others departments, and manage dependencies between teams. This report details some of the challenges they faced, how they handled them, and how ADM has helped Product Managers become more effective.''
	\item[Web link:] \url{http://ieeexplore.ieee.org/xpl/articleDetails.jsp?arnumber=4599514&queryText=agile%20manager&newsearch=true}
	\item[Full text link:] \url{http://svpma.org/eventarchives/SVPMA-07-2008-Surviving_The_Transition_To_Agile_Development-Rasmus_Mencke.pdf}
	\item[Comments:] This "Paper" guides your typical manager through transitioning into agile, this should be useful for drawing a comparrison of the command and control to the self managment of agile.
\end{description}

\section*{Paper 8}
\begin{description}
	\item[Title:] We're All in This Together 
	\item[Citation:] \cite{Together}
	\item[Abstract:] ``Being agile is hard, and some organizations experience different challenges than others. In this interview, Steve Berczuk has a conversation with Yi Lv about the change process of introducing Scrum into organizations with different cultural, geographic, and size attributes. They focus on the role line managers and Scrum Masters play in establishing a sustainable Scrum culture. They also discover that agile introduction has some universals.''
	\item[Web link:] \url{http://ieeexplore.ieee.org/xpl/articleDetails.jsp?arnumber=5604357}
	\item[Full text link:] \url{http://ieeexplore.ieee.org/stamp/stamp.jsp?tp=&arnumber=5604357}
	\item[Comments:] Here an account is made in an interview about how an entire company transitioned into agile and the obstacles that were overcome. This will help in determining what problems occur when agile is introduced.
\end{description}

\section*{Paper 9}
\begin{description}
	\item[Title:] Traditional vs Agile development a comparison using chaos theory 
	\item[Citation:] \cite{Chaos}
	\item[Abstract:] ``Agile software development describes those methods with iterative and incremental development. This development method came into view to overcome the drawbacks of traditional development methods. Although agile development methods have become very popular since the introduction of the Agile Manifesto in 2001, however, there is an ongoing debate about the strengths and weakness of these methods in comparison with traditional ones. In this paper, a new dimension for the comparison between the two methods is presented. We postulate that, since both methods are based mainly on human activity, the two methods can be modeled using Chaos Theory. Source codes that are produced by the two methods in subsequent versions are characterized by a set of software metrics. Modeling and analysis of these metrics are performed using the Chaos Theory. Initial results show that the metrics sequences of both methods are chaotic sequences. Furthermore, agile methods produce more chaotic metrics sequences. However, is being chaotic a good or a bad feature? We argue that sometimes being chaotic is not a weakness, on the contrary, it is a strength.''
	\item[Web link:] \url{http://ieeexplore.ieee.org/xpl/articleDetails.jsp?arnumber=7292582&queryText=agile%20bad&newsearch=true}
	\item[Full text link:] \url{None}
	\item[Comments:] I belive this paper looks into the chaos of the human workings and how this could be a strength in an agile enviroment. This could be useful as an argument agained the command and control method.
\end{description}

\section*{Paper 10}
\begin{description}
	\item[Title:] What Do We Know about Agile Software Development? 
	\item[Citation:] \cite{AboutAgile}
	\item[Abstract:] ``Agile software development has had a huge impact on how software is developed worldwide. We can view agile methods such as Extreme Programming (XP) and Scrum as a reaction to plan-based or traditional methods, which emphasize a "rationalized, engineering-based approach, incorporating extensive planning, codified processes, and rigorous reuse. In contrast, agile methods address the challenge of an unpredictable world, emphasizing the value competent people and their relationships bring to software development. To clarify the effectiveness of agile methods, we reviewed the agile development literature and conducted a systematic study of what we know empirically about its benefits and limitations.''
	\item[Web link:] \url{http://ieeexplore.ieee.org/xpl/articleDetails.jsp?arnumber=5222784&queryText=What%20do%20we%20know%20about%20agile%20software%20development?&newsearch=true}
	\item[Full text link:] \url{http://www.tamps.cinvestav.mx/~ertello/svam/s06-SWE-AgileSW.pdf}
	\item[Comments:] This paper delves into the Agile Methodology and its effectivness, again this will help compaer and explain agile.
\end{description}

\section*{Paper 11}
\begin{description}
	\item[Title:] When the VP is a Scrum Master, You Hit the Ground Running 
	\item[Citation:] \cite{VPMaster}
	\item[Abstract:] ``Companies adopting Scrum as a software development framework often start with a pilot project and slowly grow adoption across the organization. Unlike this typical incremental implementation of Scrum, Unisys Cloud Engineering adopted Scrum throughout the entire organization at the same time. The adoption was initiated and supported by upper management. This paper describes how the effort started, provides an assessment of why this method worked, and details lessons learned that could benefit others considering this rollout method.''
	\item[Web link:] \url{http://ieeexplore.ieee.org/xpl/articleDetails.jsp?arnumber=6005517&queryText=When%20the%20vp%20is%20a%20scrum%20master%2c%20you%20hit%20the%20ground%20running&newsearch=true}
	\item[Full text link:] \url{https://www.agilealliance.org/wp-content/uploads/2016/01/WhentheVPisaScrumMasterPresentation110805b.pdf}
	\item[Comments:] Here we see what happens when higher managment take the roll of Scrum Master whihc would be usful insite on how C and C managers handle the change.
\end{description}

\section*{Paper 12}
\begin{description}
	\item[Title:] Scrum Master Activities: Process Tailoring in Large Enterprise Projects 
	\item[Citation:] \cite{MasterActivities}
	\item[Abstract:] ``This paper explores practitioner descriptions of agile method tailoring in large-scale offshore or outsourced enterprise projects. Specifically, tailoring of the scrum master role is investigated. The scrum master acts as a facilitator for software development teams, nurturing adherence to agile practices and removing impediments for team members. But in large projects, scrum masters often work together in geographically distributed teams. Scrum masters use sprint planning to avoid development tasks that overlap team boundaries, coordinate status and effort across teams, and integrate code bases. The study comprises 8 international companies in London, Bangalore and Delhi. Interviews with 46 practitioners were conducted between February 2010 and May 2012. A grounded theory research method was used to identify that the scrum master role comprises six activities: process anchor, stand-up facilitator, impediment remover, sprint planner, scrum of scrums facilitator, and integration anchor. This systematic description of activities in scrum master teams extends our understanding of practitioner perspectives on agile process tailoring in large enterprises. Understanding these activities will help coaches guide large scale agile teams.''
	\item[Web link:] \url{http://ieeexplore.ieee.org/xpl/articleDetails.jsp?arnumber=6915249&queryText=agile%20master&newsearch=true}
	\item[Full text link:] \url{None}
	\item[Comments:] This paper explores the responsibilities of the scrum master which will help me to understand the reasoning behin the Scrum Master setup in an agile enviroment.
\end{description}

\section*{Paper 13}
\begin{description}
	\item[Title:] Manager as Scrum Master
	\item[Citation:] \cite{ManagerMaster}
	\item[Abstract:] ``Manager as Scrum Master? You can not do that! It goes against the conventional wisdom, which assumes command and control managers cannot lead and coach as Scrum Masters. However, as an important aspect in Agile change, self-managing means a management transformation from command and control to leading and coaching. This experience report will explain how one large-scale organization adopted Agile over three years, with the focus on the evolution of Scrum Master and manager role and the way they work together. It describes how the change regarding self-management was introduced and adapted, then how we have tried to sustain the change by creating consistency between the values and principles behind Agile into the organization and the management capability to practice them.''
	\item[Web link:] \url{http://ieeexplore.ieee.org/xpl/articleDetails.jsp?arnumber=6005499&queryText=agile%20master&newsearch=true}
	\item[Full text link:] \url{https://www.odd-e.com/material/2011/agile2011/Manager%20as%20SM%20in%20Agile%202011.pdf}
	\item[Comments:] Again this will help derive the difference between CandC and agile practicies.
\end{description}

\section*{Paper 14}
\begin{description}
	\item[Title:] Agile management - an oxymoron?: who needs managers anyway?
	\item[Citation:] \cite{Oxymoron}
	\item[Abstract:] ``"Self-directed team" is one of the mantras of Agile Methodologies. Self-direction means that the team's manager is relegated to a facilitator role with little or no influence over day-to-day activities. For example, Kent Beck has written that the manager of an XP project can do four things: ask for estimates on cost and results, move people around among projects, ask for status reports, and cancel the project. Agile literature in general says that managers shouldn't be directly involved in analysis, design, coding, testing or integration. They may (but only occasionally!) facilitate the process between the customer and the developers - and it would be nice if they provided food and toys to keep the team happy. It appears, then, that the agile manger is expected to hover on the fringes of a project asking a few questions and throwing in goodies - but with ultimate power (cancellation) in her hip pocket.This scenario makes one wonder. Do managers really matter to the success of an agile project? Are they superfluous? What happens when managers step over the prescribed line - does it mean that the end of Agile Methodology as we know it and as handed down by the Agile Manifesto? The panel will explore this ticklish terrain by answering the following questions.
	•Why Agile Methods and managers don't mix. Or do they?
	•What can/should managers do in an agile environment?
	•Under what conditions are managers an absolute requirement in an agile environment? (e.g. Government applications?)
	•Do good management techniques apply to both Agile and non-Agile environments?
	•Is management a dead-end profession in an Agile world?
	''
	\item[Web link:] \url{http://dl.acm.org/citation.cfm?id=949410}
	\item[Full text link:] \url{https://www.researchgate.net/publication/221321085_Agile_management_-_An_oxymoron_Who_needs_managers_anyway}
	\item[Comments:] An argument against the need for managers in an agile environment. wich will help to draw a comparison.
\end{description}

\bibliographystyle{apalike}
\bibliography{COMP150_Agile_Essay_Proposal}

\end{document}
